%******************* Named appendix A *******************
\namedappendices{A}{Named first appendix}

\checkBeginRefsection%%% Don't delete this 

Appendix material can be included here. First include a figure (fig.~\ref{fig:ap1}).

%-------------------------------------------------
\section{Appendix A - Section With Figure}
\myfigap{H}{0.5}{example-image-golden}{A golden ratio rectangle image.}{fig:ap1}	\kant[8]

%-------------------------------------------------
\section{Appendix A - Section With Table}
And, then including a table (table.~\ref{tab:ap1}).

\begin{appendixtable}[h!]
\centering
\caption{Use of \texttt{tblr} environment for full-width table - applicable to both main text 
and appendix.  Note the use of \texttt{booktabs} commands and `X' parameters to reproduce 
Table~\ref{tab:ap1}.}
\begin{tblr}{*4X}
\toprule
Number 		& Name of month 	& Days 	& Season\\
\midrule
\#7 			& July       			& 30 		& Spring\\ \cmidrule[lr]{2-4}
Multicolumn 	&\SetCell[c=3]{c} The three columns combined \\ \cmidrule[lr]{2-4}
\#8 			& August 		   	& 31 		& Summer\\
\#9 			& September 		& 30 		& Summer\\
\bottomrule
\end{tblr}
\begin{tablenotes}[flushleft]
\footnotesize
\item \hspace{-1ex} \emph{Note}: The \texttt{tablenotes} environment produces table footnotes.  
Refer to \texttt{tabularray} documentation for further details.  
\end{tablenotes}
\label{tab:ap1}
\end{appendixtable}

%-------------------------------------------------
\subsection{Appendix A Subsection}

\textcolor{gray}{Dummy text from kantlipsum[9]. Reference listing on the next page. Check it for the intended formatting.} I refer to \citep{lamport94,kopka2004guide,baczkowski1990ndsu}. 

\myalign{
y_1 & = m_1x + c_1 
\label{eq:lineA1} \\
y_2 & = m_2x + c_2 
\label{eq:lineA2} \\
y_3 & = m_3x + c_3 
\label{eq:lineA3}
}

A family of straight-line equations is presented above (\cref{eq:lineA1,eq:lineA2,eq:lineA3}).

\kant[10]

\checkEndRefsection%%% Don't delete this

%******************* END *******************