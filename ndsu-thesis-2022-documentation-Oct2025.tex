\documentclass{article}

% Source file of ndsu-thesis-2022 class documentation (ndsu-thesis-2022-documentation.tex). The images used (3 total) are stored in /figures/ subfolder. 

%--------------------------------------------------------------
\usepackage[nonewpage]{imakeidx}
\makeindex[intoc, columnseprule, columns=3, columnsep = 2.5cm]
\usepackage[top=1in,bottom=1.2in,left=1in,right=1in,letterpaper]{geometry}
%\usepackage[T1]{fontenc}
\usepackage{verbatim}
\usepackage{tocloft}
\usepackage{xparse}
\usepackage{xspace}
\usepackage[dvipsnames]{xcolor}
\usepackage{booktabs}
\usepackage{threeparttable}
\usepackage{graphicx}
\usepackage{float}
\usepackage{siunitx}
\usepackage{soul}
\usepackage[colorinlistoftodos,prependcaption,textsize=scriptsize]{todonotes}
\usepackage{tikz}
\usepackage{multicol}
\usepackage{tabularray}
\usepackage{caption}
\usepackage{enumitem}
\usepackage{chemformula}
\usepackage{sectsty}
\usepackage{textcomp}
\usepackage{wasysym}
\usepackage[hidelinks]{hyperref}

%--------------------------------------------------------------
%--------------------------------------------------------------
\graphicspath{{./figures/}}

\definecolor{sec}{HTML}{1E2761} % Indigo
\definecolor{ssec}{HTML}{7A2048} % Maroon
\definecolor{sssec}{HTML}{408EC6} % Blue

\sectionfont{\sffamily\color{sec}}  % sets colour of sections
\subsectionfont{\sffamily\color{sec!60!sssec}}  % sets colour of subsections
\subsubsectionfont{\sffamily\color{sssec}}  % sets colour of subsubsections

\setlength\lightrulewidth{0.2pt}

\makeatletter
%\newcommand\notsotiny{\@setfontsize\notsotiny\@vipt\@viipt}
%\newcommand\notsotiny{\@setfontsize\notsotiny{6}{7}}
%\newcommand\notsotiny{\@setfontsize\notsotiny{6.31415}{7.1828}}
\newcommand\notsotiny{\@setfontsize\notsotiny{6.44}{7.54}}
%\newcommand\notsotiny{\@setfontsize\notsotiny{6.693}{7.614}}
\makeatother

\newcommand\cmd[1]{\textbackslash\texttt{#1}}
\newcommand\tb\textbackslash
\definecolor{newtext}{rgb}{0.1,0.6,0.1}
\newcommand\nt[1]{\textcolor{newtext}{#1}\xspace} % annotation command
\newcommand\dt[1]{\textcolor{red}{\st{#1}}\xspace} % annotation command
\newcommand\rt[2]{{\textcolor{red}{\st{#1}}}{\textcolor{newtext}{#2}}\xspace} % annotation command
\newcommand\notes[1]{\vspace{2ex}\todo[color=green!35, inline]{#1}} % todo notes
\newcommand\ix[1]{#1\index{#1}} % shortcut for index - #1 twice to retain text
\newcommand\ixt[1]{#1\index{\texttt{#1}}} % shortcut for index - #1 twice to retain text
\newcommand\ixl[1]{#1\lowercase{\index{#1}}} % shortcut for index
\newcommand\ixm[2]{#2\index{#1!#2}} % shortcut for index - muti level
\newcommand\ixmt[2]{\index{#1!\texttt{#2}}} % shortcut for index - mutilevel
\newcommand\bt{BibTeX\xspace}% shortcut for \bt
\newcommand\blt{Bib\LaTeX\xspace}% shortcut for BibLaTeX
\setlength{\parindent}{8mm}

\newcommand\ccmd[1]{ % shortcut for centered commands
\begin{center}
\begin{tabular}{l}
#1
\end{tabular}
\end{center}
}

\newcommand\tend{\end{document}}


%\setlength{\columnsep}{3cm}

%--------------------------------------------------------------
%--------------------------------------------------------------
\title{\vspace{-1.5cm}Using the \texttt{ndsu-thesis-2022} \LaTeX\/ class---Documentation}
\author{Aaron Feickert$\,^{a}$, Jonathan Totushek$\,^{a}$, and C. Igathinathane$\,^{b,\,*}$ \\ *\,Maintainer, Bug Reports, and Enquires:  Igathinathane Cannayen (\texttt{i.cannayen@ndsu.edu})\\ {\small Github: \url{https://github.com/CIgathi/NDSU-Thesis-Class.git}}
}
\date{{\small $^{a}$\emph{Department of Mathematics, NDSU} \\ $^{b}$\emph{Department of Agricultural and Biosystems Engineering, NDSU}\\[2ex]}
18 October 2025}

%--------------------------------------------------------------
%--------------------------------------------------------------
%--------------------------------------------------------------
\begin{document}
\maketitle

\vspace{-4ex}
\hrule
\vspace{-1.5ex}
\begin{multicols}{2}
\notsotiny%\scriptsize
\tableofcontents
\end{multicols}
\vspace{-0ex}
\hrule

%--------------------------------------------------------------
%--------------------------------------------------------------
\section{Introduction}
The \texttt{ndsu-thesis-2022} \LaTeX\ is an updated version of the previous \texttt{ndsu-thesis} \ix{class} file. This class generates disquisitions intended to comply with the disquisition requirements of the North Dakota State University (NDSU) Graduate School. This class simulates the output as generated by the NDSU disquisition templates updated February 2023 (\url{https://www.ndsu.edu/gradschool/current_students/graduation/theses_dissertations_papers/disquisition_formatting}). This class is not officially endorsed by NDSU or the NDSU Graduate School, but efforts are underway toward that goal. It should be noted that several theses and dissertations were made and got approved by the Graduate School using the NDSU \LaTeX\ thesis class in the past. Since disquisition requirements are subject to change at any time, the user is advised that the most current disquisition style policies supersede this class. 

However, following the Graduate School-approved templates and collected experience from several previously approved dissertations, this \LaTeX\ class was coded to incorporate the various required features and lessons learned. To ensure compliance with all NDSU Graduate School requirements, the user is encouraged to consult the NDSU Graduate School webpage and the links provided for detailed requirements and guidance on disquisition formatting guidelines, templates, section formatting, and examples.

The bundled \ix{template} or the thesis example given (Section~\ref{example}) can be used as an easy starting point for using the class. Modification of the class file's code may result in unexpected behavior and is at the user's own risk. We recommend including additional packages and commands in the source file (*.tex) itself for the desired customization as required by the departments and the users. 

%--------------------------------------------------------------
%--------------------------------------------------------------
\section{Using and installing \LaTeX---online and desktop environments} \index{online editor} \index{standalone editor}

\LaTeX\ consists of the base software and the integrated development environment (IDE) to conveniently work with document development. Base \LaTeX\ software for different operating systems can be downloaded from this resource \url{https://www.latex-project.org/get/}. Several online (e.g., Overleaf, Kile LaTeX Editor, Authorea, Papeeria, and so on) and standalone desktop versions (e.g., TeXMaker, TeXWorks, TexShop, TeXStudio, and so on) of \LaTeX\ IDEs are available. Online editors are ``ready-to-go,'' with several templates, tutorials, and help documentation, where the user does not install the software but requires an internet connection. The desktop version requires software installation and updating (usually required after initial installation and needed only to employ new features, but it will run with existing packages). With base \LaTeX\ installed, it is possible to edit source code and compile using command line processing (e.g., cmd Window or Mac terminal).   

Resources (text and video instructions) are available on both how to use the online editor and install the \LaTeX\ desktop version of the users' choice. As \LaTeX\ is open source, most of these IDEs/editors are free, and usually, it is not necessary to spend on paid services to work with \LaTeX\ and generate a document like a graduate disquisition or journal articles. It should be noted that \LaTeX\ (manual released in 1986) is more than 35 years old and \LaTeX\ continues to be used as a high-quality free document preparation system by users of STEM fields.

%--------------------------------------------------------------
%--------------------------------------------------------------
\section{Thesis Example}\label{example}\index{example thesis}
Below is a brief example of an M.S. thesis (copy/load code in the editor, have necessary resources [class + figures], and compile for output) that includes all required and several optional elements. An attempt was made to cover most of the aspects (prefatory items, chapters, sections, tables, figures, appendices, etc.) encountered during the preparation of the disquisition using \LaTeX, therefore, the example is relatively elaborate. This example M.S. thesis code shown is included in the file named ``\texttt{ndsu-example.tex}''. In this example, the examining committee includes the Committee Chair, no Co-Chairs, and only two additional Committee Members. For this example, \textsc{Bib}\TeX\ was used to manage references, which would be included in a file named \texttt{mybib.bib} separately.

With \LaTeX, the users type some commands and texts that are specific to their thesis/dissertation, which is human-readable (source code), as shown below following a template, and compile the source to automatically generate the well-formatted NDSU thesis-style document (Fig. \ref{fig:thesis}). The benefits of using \LaTeX\ for thesis/dissertation include overall automation, open-source, freely available, vibrant society support, professional quality outcome, elegant mathematics handling, automatic bibliography management, integrated typography principles, portability among operating systems, longer life of the source code, every aspect of document preparation addressed, packages available for specialized needs, thesis/dissertation source code easily converted to journal articles with appropriate templates, and so on.        


{\scriptsize
\begin{verbatim}
%******************* START *******************
\documentclass[ms-thesis,12pt,mathdesign]{ndsu-thesis-2022}

%Refer documentation (ndsu-thesis-2022-documentation.pdf) for various options and commands

%******************* Packages, newcommands, and other customization *******************
\usepackage[style=apa,natbib=true,backend=biber]{biblatex}% works with \citep and \citet commands
\addbibresource{mybib.bib}% *.bib extension is necessary 
\renewcommand\myspacing{1.9} % 23 lines/page needs 1.9 for thesis

%******************* First and second page material *******************
\title{The Title of My M.S. Thesis}
\author{Samuel Fargo Bison}
\date{June 2023}
\progdeptchoice{Department} % Use Department (or) Program
\department{Mathematics}

\cchair{Prof. John Adams} % Use actual committee members' names 
\cmembera{Prof. Abraham Lincoln}
\cmemberb{Prof. George Washington}
\cmemberc{Prof. Theodore Roosevelt} % If 3rd not required - delete this line 
\approvaldate{12/14/2022}
\approver{Prof. James Garfield}

%******************* Front matter *******************
\abstract{This is the abstract for my thesis. \\ \emph{Abstracts for doctoral dissertations must use 350 words 
or less. Abstracts for master's papers or master's theses must use 150 words or less.}

\kant[16]} % dummy text

\acknowledgements{I acknowledge people here. \\ \emph{Acknowledgments text should be placed here.} 

\kant[15]}

\dedication{This thesis is dedicated to my cat, Mr. Fluffles.\\ \emph{This section dedicates the disquisition
to a few significant people. The text must be double-spaced and aligned center to the page.} 
\\ Which is already taken care of by this \LaTeX\ class.}

\preface{You can put a preface here. \\ \emph{This section is optional!} 

\kant[14]}

\listofabbreviations{% may use title case
AC       & alternating current \\
NDSU     & North Dakota State University \\
ZL       & zeta Level % last item does not need \\  but okay to use
}

\listofsymbols{% may use sentence case
$A$     & area (\unit{\m\squared})\\
$e$     & Euler's constant (\num{2.718281828}) \\
$R^2$   & coefficient of determination % last item does not need \\  but okay to use
}

%******************* Document start *******************
\begin{document}

%******************* First chapter - paper style *******************
\mypaperheading{The First Chapter - Paper Style - Long title of this technical paper}{This paper is planned to be 
submitted as a peer-reviewed article \ldots\ more information about the author(s),  title, \emph{journal}, to be added.}

\section{Abstract}
Paper-styled chapters will have abstracts. Abstract of this chapter goes here. \kant[1]

\section{Section ($\Rightarrow$ 1st level; Title Case; Centered; Boldface) E.g., Science is Key Period}
This is the first section of the thesis (1st level: 1.2. Section). \kant[2]

\section{Section E.g., Science is Key Period}
This is the second section of the thesis (1st level: 1.3. Section). \kant[3]

\subsection{Subsection ($\Rightarrow$ 2nd level; Title Case; Left-justified; Boldface) E.g., Science is Key Period}
This is the subsection text (2nd level: 1.3.1. Subsection). \kant[4]

\subsubsection{Subsubsection ($\Rightarrow$ 3rd level; Sentence case; Left-justified; Boldface; Italics) E.g., Science is key period}
This is the subsection text (3rd level: 1.3.1.1. Subsubsection). \kant[5]

\paragraph{Paragraph ($\Rightarrow$ 4th level; Sentence case; Left-justified; No bold; Italics)  E.g., Science is key period}
This is the subsection text (4th level: 1.3.1.1.1. Paragraph). \kant[6]

\subparagraph{Subparagraph ($\Rightarrow$ 5th level; Sentence case; Left-justified; No bold; Regular) E.g., Science is key period}
This is the subsection text (5th level: 1.3.1.1.1.1. Paragraph). \kant[7]

\section{Table and Figure}
This is the third section of the thesis (1st level: 1.4. Section). This section
illustrates the inclusion of a simple table (\cref{tab:1}) and a figure shown later.

\begin{table}[h]
\centering
\caption{Table captions go at the top of the table. This was a long caption of the table
included in the first chapter---so that we see how it breaks into another line and
has a single spacing. Usually, tables are of full width and are demonstrated subsequently.}
\vspace{-1ex}
\begin{tabular}{clr}
\toprule
Number & Month & Days\\
\midrule
\#1 & January    & 31\\
\#2 & February   & 28\\
\#3 & March      & 31\\
\bottomrule
\end{tabular}
\label{tab:1}
\end{table}	\kant[7]

Now the figure (\cref{fig:1}) illustrates an example figure from the \texttt{mwe} package.

\myfig{H}{0.525}{example-image-duck}{Caption for this example image in this first chapter.}{fig:1} 	\kant[8-9]

%******************* Second chapter - regular chapter style *******************
\myheading{The Second Chapter - Regular Style - Long title for this chapter}

Regular style chapters will not have abstracts. General information or an outline of the chapter is given
here---before breaking into sections.

\section{Excellent Results}
This is another section of the thesis (1st level: 2.1. Experimental Results). \Cref{tab:2} presents the
results in a tabular form that spans the entire width. Please note the results shown (\cref{tab:2}) 
are preliminary.

\begin{table}[ht]
\centering
\caption{Table spanning entire width (full-width) using \texttt{setlength} and
\texttt{tabcolsep}}
\vspace{-1ex}
\setlength{\tabcolsep}{3.75em}
\begin{tabular}{@{\hspace{2ex}} lccr @{\hspace{2ex}}}
\toprule
Number & Name of month & Days & Season\\
\midrule
\#4 	& April  & 30		& Spring\\
\#5 	& May    & 31		& Summer\\
\#6 	& June   & 30		& Summer\\
\bottomrule
\end{tabular}
\begin{tablenotes}[flushleft]
\item \hspace{-1ex} Note: The \texttt{tablenotes} environment produces table footnotes. 
\end{tablenotes}
\label{tab:2}
\end{table}	\kant[7-8]

\subsection{Minor Results}
This is a subsection of the thesis (1st level: 2.2. Experimental Results). 	\kant[8]
The \Cref{fig:2} is an example image with a command showing all arguments, including the optional 
caption placement. The example figure (\cref{fig:2}) is included in the \texttt{mwe} package.

\myfig[2ex]{H}{0.45}{example-image}{Caption for this example image demonstrating an optional 2ex vertical 
spacing. Compare this with a narrow caption spacing without an optional argument in \cref{fig:1}.}{fig:2}    
\kant[8]

\section{Equations}
\kant[2]

\myeqn{% shortcut for equation vertically spaced
y = (mx + c) \times \text{NCF} \times S_\text{factor} \times c_p \times M_\text{p}
\label{eq:lin}
}

\noindent where $y$ is the dependent variable, $m$ is the slope, $x$ is the independent variable, $c$ is the $y$ 
intercept, NCF is the normalized conversion factor, $S_\text{factor}$ is the scale factor, $c_p$ is the specific 
heat capacity at constant pressure ($p$, variable), and $M_\text{p}$ is the mass of a proton (p, descriptive). 
Note how variables, abbreviations, and subscripts are coded in \cref{eq:lin}. Refer Extended Thesis to know 
more about equations and shortcuts. 

\section{Schemes}
\kant[2]

The regular way of coding a scheme:

\begin{scheme}
\centering
\includegraphics[width=0.4\textwidth]{LampFlowchart}
\caption{Flowchart of controls of light bulb---A scheme}
\label{sc1}
\end{scheme}
%

\kant[9]\vspace{-1.5ex}

\mysch[2ex]{h!}{0.48}{LampFlowchart}{Caption for this example image demonstrating an optional 2ex vertical spacing. 
Compare this with a narrow caption spacing without an optional argument in \cref{fig:2}.}{sc2}

\kant[2]\kant[9]

The (\cref{sc1,sc2}) are good. And, stating this differently, all the \Cref{sc1,sc2,sc3} are good too. Note that 
\Cref{sc3} is in landscape mode.

\myschls[1ex]{p}{0.65}{LampFlowchart}{Landscape scheme---Flowchart of controls of light bulb. Optional 2ex vertical 
spacing was used.}{sc3}

\section{Some References}
Referring to all entries in the ``\texttt{mybib.bib}'' file to generate the citations here and the listing 
using the \texttt{\textbackslash citep\{\ldots\}} ``natbib'' command (cite parenthesis) \citep{texbook,
lcompanion,latex2e,knuth1984,lesk1977,amsthm2017,calvo2004using,cannayen2011latex,kopka2004guide,notso2021,
bari2016identification}.

The same using \texttt{\textbackslash citet\{\ldots\}} command (cite text) in the running text as: The authors 
\citet{texbook,lcompanion,latex2e,knuth1984,lesk1977,amsthm2017,calvo2004using,cannayen2011latex,kopka2004guide,
notso2021,bari2016identification} have something to do with \LaTeX. For most bibliography citations and list 
creation, these two commands are sufficient.

%******************* Bibliography handling *******************
\makerefs %For individual chapter references - command should be inside refsection environment

%******************* Named appendix A *******************
\namedappendices{A}{Named first appendix}
Appendix material can be included here. First, a paragraph of text and then an example figure (fig.~\ref{fig:ap1}).

\section{Appendix A - Section With Figure}
\kant[9]
\myfigap{H}{0.5}{example-image-golden}{A golden ratio rectangle image.}{fig:ap1}	\kant[8]

\section{Appendix A - Section With Table}
And, then including a table (table.~\ref{tab:ap1}).

\begin{appendixtable}[h!]
\centering
\caption{Use of \texttt{tblr} environment for full-width table - applicable to both main text and appendix.  
Note the use of \texttt{booktabs} commands and `X' parameters to reproduce Table~\ref{tab:2}.}
\begin{tblr}{*4X}
\toprule
Number 	& Name of month 	& Days 	& Season\\
\midrule
\#7 			& July       & 30 		& Spring\\ \cmidrule[lr]{2-4}
Multicolumn 	&\SetCell[c=3]{c} The three columns combined \\ \cmidrule[lr]{2-4}
\#8 		& August 		   & 31 		& Summer\\
\#9 		& September 	& 30 		& Summer\\
\bottomrule
\end{tblr}
\begin{tablenotes}[flushleft]
\item \hspace{-1ex} Note: The \texttt{tablenotes} environment produces table footnotes.  
Refer to \texttt{tabularray} documentation for further details.  
\end{tablenotes}
\label{tab:ap1}
\end{appendixtable}

\subsection{Appendix A Subsection}
\kant[10]

%******************* Named Appendix B *******************
\namedappendices{B}{Named second appendix}
Appendix material can be included here. First include a figure (fig.~\ref{fig:ap2}).

\section{Appendix B - Section With Figure}
\kant[9]
\myfigap[0.5ex]{H}{0.6}{example-grid-100x100pt}{A $10 \times 10$ grid of different concentric colors.}{fig:ap2}

\section{Appendix B - Section With Table}
Now coding another appendix table (table.~\ref{tab:ap2}) that spans the entire width using the manual method 
(using `tabcolsep' command; and `resize' command to fit large tables).

\begin{appendixtable}[h]
\centering
\caption{Squares and cubes in named appendix table using \texttt{siunitx} and \texttt{tabularray} 
packages.}
\begin{tblr}{X X[c] X[r] X[1.5,r]}
\toprule
Number & Square        & Cubes          & Fourth power\\
\midrule
11 	   & 121   			       & \num{1331} 		   & \num{14641}\\
22 	   & 484  			        & \num{10648}		   & \num{234256}\\
333 	  & \num{110889}  & \num{36926037}	& \num{12296370321}\\
\bottomrule
\end{tblr}
\label{tab:ap2}
\end{appendixtable}

\subsection{Appendix B Subsection}
\kant[11]

\closeappendices   % Refer to documentation Table 3 for proper closing 

\end{document}
%******************* END *******************
\end{verbatim}
}

\begin{figure}[p]
\centering
\vspace{-0.65ex}
\hspace*{-0.6in}
\includegraphics[width=1.2\textwidth]{fig-thesis.pdf}
 \caption{Automatically formatted output sample pages of the example thesis according to NDSU Graduate School requirements. Several pages were skipped to show the overall outcome and the source code.}
\label{fig:thesis}
\end{figure}

The example thesis code, when compiled, will produce the output shown in Figure~\ref{fig:thesis}. The example source code serves as a lean template with meaningful and dummy text. The citations in the text and reference listing were automatically generated according to the selected reference style.

A lightweight source file named ``\texttt{ndsu-sandbox.tex},'' which can be used to try out things conveniently in the actual NDSU thesis environment, was also included in the package folder. Things tested here (including the bibliography) can be readily inserted into the original thesis/dissertation document. In addition, another extended file named ``\texttt{NDSU-Thesis-Extended.tex}'' (URL: \url{https://github.com/CIgathi/NDSU-Thesis-Class}) containing several additional comments (listing the various options of the class) and features was made available as a supporting document. The code below (\texttt{ndsu-example.tex}) will also work when directly extracted (selected and copied) and compiled. Most of the requirements of the students for disquisition will be covered with these two example source code files. 

%--------------------------------------------------------------
%--------------------------------------------------------------
\section{Documentclass Options}
These are the \ix{options} passed to the \ix{documentclass} command while calling the class. These options essentially affect the whole document, and a \ix{default behavior} (no options specified, shown below) was also valid.
\ccmd{\cmd{documentclass\{ndsu-thesis-2022\}}}
The default behavior with no \texttt{[options]} specified, as shown above, produces a Ph.D. dissertation in \qty{12}{pt} font size with auto-numbered headings and justified text in Computer Modern font. However, the command:\vspace{-2ex}\ccmd{\cmd{documentclass[ms-thesis,11pt,nonumber,nojustify,draft,showframe,times]\{ndsu-thesis-2022\}}} produces an M.S. thesis in \qty{11}{pt} font size unjustified paragraphs, text with unnumbered headings in draft mode, in Times Roman font, and shows the frame using the set margins. The order in which these options are passed does not matter. 

%--------------------------------------------------------------
\subsection{Disquisition degree and type}
\label{degtype}
One of the important options of the class is the degree type. By default, this class assumes the document is a Ph.D. dissertation. The other types of available degree and disquisition types are given in Table~\ref{degree}. 

\begin{table}[h!]
\centering
\caption{Options for degree and disquisition types}
\vspace{-1.5ex}
\begin{tabular}{lll}
\toprule
Option & Degree & Disquisition type \\
\midrule
\texttt{[\ixm{degree}{phd}]} & Ph.D. & Dissertation (default) \\
\texttt{[\ixm{degree}{ms-thesis}]} & M.S. & Thesis \\
\texttt{[\ixm{degree}{ms-paper}]} & M.S. & Paper \\
\texttt{[\ixm{degree}{ma-thesis}]} & M.A. & Thesis \\
\texttt{[\ixm{degree}{ma-paper}]} & M.A. & Paper\\
\bottomrule
\end{tabular}
\label{degree}
\end{table}

%--------------------------------------------------------------
\vspace{-1.5ex}
\subsection{Collection of all options of documentclass}

The various documentclass options, which are related to font size, section numbering, text justification, showing helpful frames and grids, font styles, degrees and types, and bibliography handling, as well as the defaults (shown in blue), are listed in Table~\ref{options}. If any other undefined options or mistyped options are ignored, the defaults will be used for the output. Details of the options are described subsequently. 

\vspace{-1ex}
\begin{table}[ht]
\small
\centering
\caption{List of all documentclass options and the defaults already loaded}
\vspace{-1.5ex}
\ttfamily
\setlength{\tabcolsep}{0.18in}
\begin{tabular}{@{\:\:}l  l  l  l  l}
\toprule
10pt & 11pt & \textcolor{blue}{{12pt}  (d)} & nonumber & \textcolor{blue}{{numbered} (d)} \\
chapternumber & nojustify & draft & showframe  & showgrid\\
bookman & charter & gentium & kpfonts & libertine\\
mathdesign & mathptmx & mlmodern & newcent & newpx\\
newtx & palatino & tgtermes & times & tgbonum \\
tgpagella & tgschola & utopia & clearsans & cmbright\\
firasans & helvet & kurier & lxfonts & sansmathfonts\\
\textcolor{blue}{{computermodern} (d)}  & chapterrefs & \textcolor{blue}{{phd} (d)} & ms-thesis & ms-paper \\
ma-thesis & ma-paper & chaptersbib & subfileref & \\
\bottomrule
\multicolumn{5}{l}{\hspace{-4.8mm} \footnotesize {\normalfont \emph{Note}:\:\textcolor{blue}{Option \texttt{(d)}} - default options already loaded (need not specify them in the documentclass)}}
\end{tabular}
\label{options}
\end{table}

%--------------------------------------------------------------
\subsection{General requirement for thesis---formats, margins, and spacings}
The most general and mandatory requirement for the thesis is adherence to the rules set by the school. These include the (i) formats (headings, captions, general test, and appropriate cases and style of headings and text), and (ii) margins and spacings (margins of page [1 inch all around]; double-spacing for general text, and single spacing for captions and table contents; paragraph indent [1/2 inch]; spacing around all elements [tables, figures, schemes] should be the same as the double-spacing followed with the general text). 

The NDSU class is developed to take care of these requirements; however, as \LaTeX\ algorithm typesets elements automatically based on the available space and introduces additional space (as seen below the Table~\ref{options} and the heading of this section). This additional space can be corrected using \cmd{vspace\{\}} command with $-$ve values to suit the requirement. The document "ndsu-thesis-vertical-spacing"\index{vertical spacing} (URL: \url{https://github.com/CIgathi/NDSU-Thesis-Class}) added in the package exclusively addresses this issue. It will also be useful to utilize the documentclass \texttt{showframe} option (See fig.~\ref{fig:frame}) to visualize the margin for page elements. The \texttt{showframe} option will be helpful, specifically for the landscape page elements, for correct resizing so that everything fits as it should. 

When there is no room to fit an element (table or figure) in the given space, it will be floated to the next page, leaving a blank space on the current page, such blank spaces are ``okay'' to have, as these are unavoidable. Sometimes, it is possible to move the text after the float element and fill this blank space, as long as the content allows such a move (e.g., major headings or other floats cannot be moved to fill this space). 





%--------------------------------------------------------------
%--------------------------------------------------------------
\section{Preamble Information}

%--------------------------------------------------------------
\subsection{General information - packages and shortcuts}
If your disquisition requires the use of additional \LaTeX\ packages, macro files, or other commands, include them in the preamble. Packages such as \texttt{\ix{natbib}} (author-year and number style citations) or other competent reference handling packages and their options can be loaded. Similarly, mathematical \ix{theorem environment}-related commands (theorem, corollary, lemma) through \cmd{newtheorem} of \texttt{amsthm} package, \texttt{caption} package setups through \cmd{captionsetup[\emph{type}]\{\emph{options}\}}, where \emph{\texttt{type}} = table, figure, or subfigure, and other shortcuts for repetitive longer commands or text can be defined in the preamble. As these are specific to the users and the requirements vary with the users of different specializations, these were not included in the class. Therefore, suitable packages, commands, and shortcuts can be defined by the users.

%--------------------------------------------------------------
\subsection{Dissertation front pages}
Before issuing the \cmd{begin\{document\}} command, several pieces of dissertation prefatory (preamble) information are available. 


%--------------------------------------------------------------
\subsubsection{Title}
Include the \ix{title} of the disquisition using the \cmd{title\{\ldots\}} command. This is required. The title is optimized for a title text that is 2 lines long. For others, the \cmd{vspace\{\ldots\}} command with +ve or -ve values may be issued as the \cmd{title} argument.  

%--------------------------------------------------------------
\subsubsection{Author}
Include the full name of the disquisition \ix{author} using the \cmd{author\{\ldots\}} command. This is required.

%--------------------------------------------------------------
\subsubsection{Major \ix{department}/\ix{program} choice}
Specify whether it is ``Department'' or ``Program'' that is applicable using the \cmd{progdeptchoice\{\ldots\}} command. This will ultimately produce ``Major Department:'' or ``Major Program:'' based on the input choice. This is required.

%--------------------------------------------------------------
\subsubsection{Department or program}
Include the name of the major \ix{department} or \ix{program} using the \cmd{department\{\ldots\}} command. This is required.

%--------------------------------------------------------------
\subsubsection{Degree option}
If the major department or program has a \ix{degree} option, indicate this using the \cmd{degreeoption\{\ldots\}} command. This is optional.

%--------------------------------------------------------------
\subsubsection{Date}
Include the date of the final examination using the \cmd{\ix{date}\{\ldots\}} command. The accepted format of this date is \textit{month year} as: \cmd{date\{October 2022\}}. This is required.

%--------------------------------------------------------------
\subsubsection{Examining committee}
Include the Chair (or Co-Chairs) and members of the examining committee using separate commands. The \cmd{\ix{cchair}\{\ldots\}} command is used to indicate the committee Chair. Use \cmd{\ix{cochairZ}\{\ldots\}} to indicate any committee Co-Chair members, where \texttt{Z} is \texttt{a} or \texttt{b}. This class does not support more than two Co-Chairs and four Committee Members. Use the \cmd{cmemberX\{\ldots\}} to indicate other committee members, where \texttt{X} is \texttt{a}, \texttt{b}, \texttt{c}, or \texttt{d}. Use only as many of these commands as needed to list all committee members.

%--------------------------------------------------------------
\subsubsection{Approval information}
Use the \cmd{approvaldate\{\ldots\}} command to include the full date of \ix{disquisition approval} (i.e., month/date/year). This date is generally the date the thesis was approved by the Department Chair following the defense, after the approval (usually electronically) of all committee members. Use \cmd{approver\{\ldots\}} to include the Department Chair who approved the disquisition. Both commands are required.

%--------------------------------------------------------------
\subsection{Dissertation front matter}

\subsubsection{Abstract}
Use the \cmd{\ix{abstract}\{\ldots\}} command to include the disquisition abstract. Abstracts for doctoral dissertations must use 350 words or less. Abstracts for master's papers or master's theses must use 150 words or less. This is required.

%--------------------------------------------------------------
\subsubsection{Acknowledgements}
If the disquisition includes \ix{acknowledgements}, include them using the \cmd{acknowledgements\{\ldots\}} command. This is optional.

%--------------------------------------------------------------
\subsubsection{Dedication}
If the disquisition includes a dedication, include it using the \cmd{\ix{dedication}\{\ldots\}} command. This is optional.

%--------------------------------------------------------------
\subsubsection{Preface}
If the disquisition includes a \ix{preface}, include it using the \cmd{preface\{\ldots\}} command. This is optional. The NDSU guidelines state:
\begin{quote}
``\emph{The Preface can provide an autobiographical account of how the disquisition came to be, or include a significant quote that drove your research. Follow the General Requirements for font, spacing, and page numbers for prefatory materials.}''
\end{quote}

%--------------------------------------------------------------
%--------------------------------------------------------------
\section{Automatic Components}
Several \ix{automatic components} will be generated, as a part of the front matter, based on the source code of the dissertation, and are briefly described. Based on the department, requirement, and style of the thesis, some of the items, such as lists of abbreviations, symbols, and appendix tables and figures (Secs.~\ref{sloa}--\ref{sloat}) may be dropped from the coding.

%--------------------------------------------------------------
\subsection{Table of contents}
The \ix{table of contents} (\ix{TOC}) gets automatically generated with entries up to three levels of sections (\cmd{my...heading}, \cmd{section}, and \cmd{subsection}). The dissertation may have further levels of sections, but they are not shown in the TOC.

%--------------------------------------------------------------
\subsection{List of \ix{table}s, \ix{figure}s, and \ix{scheme}s}
The \ix{list of tables} (\ix{LOT}), \ix{list of figures} (\ix{LOF}), and \ix{list of schemes} (\ix{LOSH}) will be generated based on the table, figure, and scheme full captions in the \texttt{table}, \texttt{figure}, and \texttt{scheme} environments. New commands for handling figures such as \cmd{myfig\{1+5 arguments\}}, \cmd{myfigls\{1+5 arguments\}}, \cmd{mysch\{1+5 arguments\}}, and \cmd{myschls\{1+5 arguments\}} with their own \texttt{[optional]} argument to adjust the position of the caption with respect to figure and scheme elements were defined.

%--------------------------------------------------------------
\subsection{List of abbreviations}\label{sloa}
The collection of abbreviations used in the dissertation can be made into a \ix{list of abbreviations} (\ix{LOA}) using the \cmd{listofabbreviations\{\ldots\}} command. This collection should be alphabetized before coding. Lowercase may be used for the definitions unless items are proper names. This will be a two-column tabular entry. A two-entry example of the LOA code and the output is shown below:
{\small
\begin{tabbing}
xxxx\= \kill
\> \cmd{listofabbreviations\{} \\
\> \texttt{AC \& alternating current\tb\tb} \\
\> \texttt{NDSU \& North Dakota State University\tb\tb}\\
\> \texttt{ZL \& zeta level}\}
\end{tabbing}
}
{\vspace*{-0.8in}\hspace*{3.3in}\includegraphics[width=0.55\textwidth]{fig-LOA.pdf}}

%--------------------------------------------------------------
\subsection{List of symbols}\label{slos}
The collection of all technical symbols used in the dissertation, usually coded in ``math'' mode, can be made into a \ix{list of symbols} (\ix{LOS}) using the \cmd{listofsymbols\{\ldots\}} command. This collection should be alphabetized before coding, and math mode should be used as required. Again, lowercase may be used for definitions unless items are proper names. This will be a two-column tabular entry. A three-entry example of LOS also using the \texttt{siunitx} package and the output is shown below:
{\small
\begin{tabbing}
xxxx\= \kill
\> \cmd{listofsymbols\{} \\
\> \texttt{\$A\$ \& area (\tb si\{\tb m\tb squared\}) \tb \tb} \\
\> \texttt{\$c\$ \& speed of light (\tb SI\{299.792\}\{\tb km\tb per\tb s\})\tb \tb} \\
\> \texttt{\$R\^{}2\$ \& coefficient of determination}\}
\end{tabbing}
}
{\vspace*{-0.8in}\hspace*{3.3in}\includegraphics[width=0.55\textwidth]{fig-LOS.pdf}}

%--------------------------------------------------------------
\subsection{List of appendix tables and figures}\label{sloat}
The \ix{list of appendix tables} (\ix{LOAT}) and \ix{list of appendix figures} (\ix{LOAF}) will be generated based on the appendix table and figure full captions in the \texttt{appendixtable} and \texttt{appendixfigure} environments. New commands such as \cmd{myfigap\{5 arguments\}} and \cmd{myfigapls\{5 arguments\}} plus one \texttt{[optional]} argument for adjusting the caption placement for regular and landscape figures, were defined.

The \cmd{\ix{closeappendices}} command should be issued at the end of the last appendix, which ensures the automatic creation of the LOAT and LOAF when the appendices have tables and figures. If the appendices had only tables or only figures, then the commands \cmd{\ix{closeappendixtables}} or \cmd{\ix{closeappendixfigures}} should be used to avoid blank entries, and if no tables are figures used in appendices then these commands should not be used or commented out and other details are shown in Table~\ref{appclose}. 

%--------------------------------------------------------------
%--------------------------------------------------------------
\section{Basic Components and Commands}
\subsection{Beginning the document}
After including the necessary preamble information, use \cmd{begin\{document\}} to start the \ix{document}. This command automatically generates the necessary cover pages and other automatic components. The usual \cmd{\ix{maketitle}} command should not be used, as it was already issued in the class.

%--------------------------------------------------------------
%--------------------------------------------------------------
\subsection{Headings}\label{heading}\index{heading}
Major headings (e.g., chapters) are issued using the \cmd{myheading\{\ldots\}} command. This command supersedes the usual \cmd{\ix{chapter}} command, which should not be used. The following shows the hierarchy of headings:
\ccmd{
\cmd{\ixm{heading}{myheading}\{\ldots\}} \hspace{0.8in} Produces all-caps chapter headings automatically - 0th level \index{myheading}\\
\cmd{\ixm{heading}{mypaperheading}\{\ldots\}\{\ldots\}} \hspace{0.07in} Produces all-caps paper chapter headings with footnote - 0th level \index{mypaperheading}\\
\cmd{\ixm{heading}{section}\{\ldots\}}  \hspace{0.94in} Produces centered, \textbf{bold} headings (use title case) - 1st level \index{section}\\
\cmd{\ixm{heading}{subsection}\{\ldots\}} \hspace{0.72in} Produces left-aligned, \textbf{bold} headings (use title case) - 2nd level \index{subsection}\\
\cmd{\ixm{heading}{subsubsection}\{\ldots\}}\hspace{0.55in}  Produces left-aligned, \textbf{bold}, \emph{italic} headings (use sentence case) - 3rd level \index{subsubsection}\\
\cmd{\ixm{heading}{paragraph}\{\ldots\}}\hspace{0.84in}  Produces left-aligned, \emph{italic} headings (use sentence case) - 4th level \index{paragraph} \\
\cmd{\ixm{heading}{subparagraph}\{\ldots\}}\hspace{0.62in}  Produces left-aligned, regular headings (use sentence case) - 5th level \index{subparagraph} 
}

Each \cmd{myheading\{\ldots\}} or \cmd{mypaperheading\{2 args\}} command starts a new page and entry in the table of contents.  The regular chapter command is simple and takes one argument, which is the title of the chapter as \cmd{myheading\{title\}}. The paper-styled chapter takes three arguments that address the title and footnote as \cmd{mypaperheading\{title\}\{\ix{footnotetext}\}}. The paper-style chapters have footnotes, and the title has a numbered footnote mark. The \texttt{title} is common to both styles and will be rendered as all-caps irrespective of the input. The automatically generated footnote mark, an integer sequentially incremented, is associated with the footnote text, which is displayed at the end of the page. The footnote text usually details the publication information with the roles of authors (refer to Grad School information for further details). 

Based on the latest instruction, only the ``previously published papers'' should have the footnotes, which detail the citation of the paper and the roles of the authors, and other ``papers in preparation/submission'' need not have the footnotes. To code those papers, the users can simply go for \cmd{myheading\{\ldots title\ldots\}}, where the title is a regular paper-style title.  
	 
In general,  the paper-styled chapter requires an ``Abstract'' section, while the regular chapter (e.g., Introduction, Review of Literature, etc.) does not. The class is coded to produce a consistent space between the title and the text (or section) below the title; however, when necessary \cmd{vspace\{+ve or -ve\}} can be issued before the plain introductory text or section command to adjust this vertical space.  The \cmd{showgrid} documentclass option, if required, can be used to judge the vertical spacing. 

Instances of \cmd{\ix{subsubsection}\{\ldots\}} and higher levels do not appear in the TOC, though they are included in the document.  Other than the chapter headings,  the rest of the item headings should be coded by the user manually with appropriate capitalization (title and sentence cases).
 
%--------------------------------------------------------------
%--------------------------------------------------------------
\subsection{Dummy text and images}\index{dummy text} \index{dummy images}
Users will be curious to see what their thesis/dissertation will look like quickly, without using the actual texts and figures. The class comes loaded with necessary packages such as \texttt{kantlipsum}\index{kantlipsum} (for dummy text---philosophical prose paragraphs in English) and \texttt{mwe}\index{mwe} (``minimal working examples'' for dummy images). These will help visualize the whole document (fonts, spacing, and layout) with minimal effort, and this is a common practice among typesetters to use such dummy text and images. Commands from these packages are used in the thesis example (Sec.~\ref{example}).

Commands like \cmd{kant[1]} or \cmd{kant[4-8]} will produce single or multiple dummy text paragraphs. Similarly, dummy images included in the \texttt{mwe} package can be accessed using their specific names and can be used as the image argument in the \cmd{includegraphics} command, which means that the user need not use their images. Some of the commonly used examples images are: \texttt{example-image}, \texttt{example-image-a}, \texttt{example-image-b}, \texttt{example-image-c}, \texttt{example-image-16x10}, \texttt{example-image-golden}, \texttt{example-image-pl} \texttt{ain}, \texttt{example-image-duck}, \texttt{example-image-empty}, and \texttt{example-} \texttt{grid-100x100pt}\index{example images}. Refer to the documentation of these packages for further information.

%--------------------------------------------------------------
%--------------------------------------------------------------
\subsection{Tables}
Different kinds of tables, such as simple table without caption (\texttt{\ixm{table}{tabular}}), table with \ixm{table}{caption} (\texttt{table}), table with \ixm{table}{footnote} (\texttt{\ixm{table}{threeparttable}}), table spanning entire text width (\texttt{\ixm{table}{tblr}}), table spanning multiple pages (\texttt{\ixm{table}{longtable}}), and table in \ixm{table}{landscape} page (\texttt{\ixm{table}{pdflscape}}) can be coded following the documentation of respective packages, and no shortcuts were defined as they were not practical. Using \texttt{\ixm{table}{booktabs}} package, the professional quality tables (Sec.~\ref{degtype}) can be created. Examples of using these commands can be found in the example and/or extended example theses (URL: \url{https://github.com/CIgathi/NDSU-Thesis-Class}) of the class. The guiding principle is to have a compact table based on the number of columns and avoid excessive space between columns, as well as font sizes not too small when more columns of information. 

%--------------------------------------------------------------
\subsubsection{Tables with fewer columns}
Compact tables with fewer columns are common and readily made by the common \texttt{tabular} environment. Where the columns will be based on the width of the widest entries, and the columns will be naturally spaced and result in a compact table with the total width usually less than the textwidth. No special action is necessary to make these tables. Tables of fewer columns and narrower widths need to be positioned on the page consistently. Either all of them are left-justified or centered. Footnotes corresponding to the width of the table can be coded through the \cmd{multicolumn\{no of cols\}\{lcr\}\{text\}}\tb\tb for single line items or \cmd{multicolumn\{no of cols\}\{p\{dimension\}\}\{text\}} for footnotes that run like a paragraph. The width of the footnote is controlled by the amount of text or the dimension of the paragraph (refer to the ``\texttt{NDSU-Thesis-Extended.tex}'' (URL: \url{https://github.com/CIgathi/NDSU-Thesis-Class}) for example codes). 
 

%--------------------------------------------------------------
\subsubsection{Full-width tables---\texttt{tblr} environment from \texttt{tabularray} package}\index{table!full-width}
When more columns are present in the table and can fit in the textwidth, it is better to code as automatic full-width tables. A quick and efficient method of creating tables that automatically span the entire textwidth is the use of \texttt{tblr} (short for \texttt{top-bottom-left-right} or \texttt{tabularray}) environment instead of the usual \texttt{tabular} inside the \texttt{table} environment. The \texttt{tblr} environment (from \texttt{tabularray} package) uses a special column justification code \texttt{X} (default \& options). This \texttt{X} code allots fixed column width based on the number of columns specified (default) and customizes individual columns' proportional width using coefficients with options. The \texttt{tblr} also takes the usual \texttt{l, c, r,} and \texttt{p} justification codes as well as the commands of \texttt{booktab} in the usual manner (e.g., \texttt{X[2.6,c]} is a center-justified column with 2.6 times the width of a basic column). The \texttt{tblr} environment is modern and has several special features, such as specifying exclusive \texttt{math} mode column (no \$ symbols required for individual items); SI~units features; colored rows, cells, and lines; and so on (see \texttt{tabularray} package documentation). Footnotes in  \texttt{tblr} environment can be easily coded using \texttt{tablenotes} nested environment. 

%--------------------------------------------------------------
\subsection{Figures and schemes}\label{figs}

The \textbf{figures} usually are pictures, photographs, drawings, maps, illustrations of samples, fields, instruments, structures, methods; graphs or plots of measurements, results; or anything graphically depicted to convey the thoughts. However, \textbf{schemes} are systematic plans for implementing an idea or concept, usually used to depict a process flow and the steps involved, and often involve ``arrows'' connecting one step to the next. Examples of schemes are chemical process diagrams, sets of chemical reaction pathways, flowcharts (process and computer algorithms), electrical circuits, block diagrams connected by arrows, and so on.  

It should be noted that the long format coding of figures using ``\texttt{figure}'' environment with \\\cmd{\ixm{figure}{includegraphics}\{\ldots\}} \ixm{figure}{centering}, \ixm{figure}{resize}, \ixm{figure}{caption} and \ixm{figure}{label}s is the direct approach and is always available to the users. Similarly, the schemes are coded using ``\texttt{scheme}'' environment. By default, the schemes are labeled as \textbf{\ix{Schematic}} in their caption. 

%--------------------------------------------------------------
\subsubsection{Shortcuts for figures and schemes---with direct and optional arguments}\label{figureshortcut}
However, for convenience, a set of single command shortcuts, with five arguments plus one optional, is defined. These commands specify (1) [optional] vertical placement of the caption (moving it up and down with respect to the bottom of the figure, especially for images with excessive or too less whitespace), (2)~placement, (3) size factor, (4) input file, (5) caption, and (6) label were defined to produce figures (regular and landscape). The default caption's \texttt{aboveskip} is \verb|0ex|, and this value can be changed using the optional argument. Following are examples of figure shortcuts for regular and landscape figures and schemes without and with the optional argument. \index{figure!myfig} \index{figure!myfigls} \index{figure!landscape} \index{myfig}\index{myfigls}

\begin{flushleft}
\hspace{-0.2cm}
\begin{minipage}{0ex}
\begin{verbatim}
 \myfig{ht}{0.7}{image1.jpg}{Caption for this regular figure}{fig:1}
 \myfig[1.5ex]{ht}{0.7}{image1o.jpg}{Figure caption with placement option}{fig:1o}
 \mysch{ht}{0.7}{image1.jpg}{Caption for this regular scheme}{sch:1}
 \mysch[1.5ex]{ht}{0.7}{image1o.jpg}{Scheme caption with placement option}{sch:1o}
 
 \myfigls{p}{1.32}{image2.pdf}{Caption for this landscape figure}{fig:2ls}
 \myfigls[2ex]{p}{1.31}{image3.pdf}{Landscape figure caption with placement option}{fig:3ls}
 \myschls{p}{1.32}{image2.pdf}{Caption for this landscape scheme}{sch:2ls}
 \myschls[2ex]{p}{1.31}{image3.pdf}{Landscape scheme caption with placement option}{sch:3ls}
 \end{verbatim}
\end{minipage}
\end{flushleft}
%------------------------
\index{mysch}\index{myschls}

These shortcuts (and regular float environments as well) are automatically included in the LOT, LOF, and LOSH that appear after the TOC. Sometimes, excessive spaces were observed above and below the figures and tables (floating elements) with respect to the text around. The use of vertical spacing (+ve or -ve; e.g., \cmd{vspace\{4pt\}} and \cmd{vspace\{-6pt\}}) around the floating elements can help in the adjustment of their placements. \index{figure!placement} The \ix{vertical spacing} commands can be issued before and after these environments (as required) to fix the spacing. 

Coding tables, figures, and schemes will automatically create the LOT, LOF, and LOSH. A similar approach can be used for appendix figures (Sec.~\ref{apfigtab}); however, for simplicity, appendix schemes (which can be coded as appendix figures) are not included in the class.

%--------------------------------------------------------------
\subsection{Captions}
Because of the way spacing is handled, \ix{caption}s in \texttt{table} environments must appear at the top of the table, while captions in \texttt{figure} environments must appear at the bottom of the figure. NDSU style follows left-justification and bold font (only for the labels) for both captions. If you use both \cmd{caption} and \cmd{label} commands in these environments, the \cmd{caption} command must come before the \cmd{label} command to ensure the environment is numbered correctly. The captions are coded in such a way that shorter ones are centered and longer ones are left-justified; however, as default, the table captions are left-justified and can be changed,  as outlined subsequently, to fit the requirement.  The style of the caption can be basic or specific to the department, usually following the parent technical society's leading journal or style guide. The style of labeling (e.g., regular \emph{vs} bold \emph{vs} italic, naming: Fig. \emph{vs} fig. \emph{vs} Figure, etc.) can also be adopted from the leading journal. The various options available for caption using the \texttt{caption} package (already loaded) can be set through the \cmd{\ix{captionsetup}\{\ldots\}} command. The common options are \texttt{position, skip, belowskip, aboveskip, font, labelfont, labelsep, singlelinecheck, format, justification,} and so on.  

%--------------------------------------------------------------
%--------------------------------------------------------------
\subsection{Equations}
About handling \ix{equation}s, the NDSU's guidelines state ``When coding equations, the guidelines call for the equation to be center-aligned, with the equation number aligned flush with the right margin.'' The strong suit of \LaTeX\ is the professional manner it typesets the equations and mathematical elements. Show below is the distance formula that was defined and referred (eq.~\ref{eq:1dist}), which satisfies NDSU's guidelines:
\begin{equation}
\text{Distance formula:} \quad d = \sqrt{(y_2-y_1)^2+(x_2-x_1)^2}
\label{eq:1dist}
\end{equation}
\noindent where, $d$ is the distance; and $x_1, y_1, x_2$ and $y_2$ are the coordinates of the two points. The equations can be displayed (e.g., eq.~\ref{eq:1dist}) produced by \$\$\ldots\$\$ or \textbackslash[\ldots\textbackslash] or \texttt{equation} environment; and the inline as: $ax^2 + bx + c = 0$ produced by \$\ldots\$ or \textbackslash(\ldots\textbackslash). There exist several other commands available to produce equations through several packages (e.g., \texttt{align, array, eqnarray, gather, split}), and any imaginable mathematical information can be coded. Also, \LaTeX\ supports a huge list of symbols (Refer: The Comprehensive \LaTeX\ Symbol List; 18,150 symbols; 422 pages; \url{https://tug.ctan.org/info/symbols/comprehensive/symbols-a4.pdf}) that can be used in general text or equations.

%--------------------------------------------------------------
\subsubsection{Shortcuts for equations---properly spaced vertically}\label{eqnshortcuts}
\LaTeX\/ engine sets an extra little vertical spacing around equations and non-textual elements (e.g., tables, figures) to make them stand out from the regular text, which is an expected and normal behavior. However, the NDSU guidelines require no additional vertical spacing (same double line spacing) around equations. This additional spacing can be manually corrected by using a $-$ve valued argument in the \cmd{\ix{vspace}\{\ldots\}} command before and after the elements as required (see Sec.~\ref{spacearound} for details). 

For convenient and automatic correct vertical spacing around equations, the following direct and starred versions of shortcut commands were coded for use in the class: 
 \cmd{myeqn\{\:\}},  \cmd{myeqn*\{\:\}}, 
 \cmd{myfraceqn\{\:\}},  \cmd{myfraceqn*\{\:\}}, 
 \cmd{myalign\{\:\}},  \cmd{myalign*\{\:\}}, 
 \cmd{myfracalign\{\:\}},  \cmd{myfracalign*\{\:\}}, 
 \cmd{mygather\{\:\}}, \\ \cmd{mygather*\{\:\}}, 
 \cmd{myfracgather\{\:\}},  \cmd{myfracgather*\{\:\}}, 			
 \cmd{mydisp\{\:\}}, and 
 \cmd{myfracdisp\{\:\}}.  The arguments in these commands are the actual codes of the equation(s) without their environment (already included) in these shortcuts (shown below). Label commands can be included in unstarred shortcuts as usual.  

\begin{tabbing}
xxx\= xxxxxxxxxxxxxxxxxxxxxx\=xxxxxxxxxxxxxxxxxxxxxxx\= xxxxxxxxxxxxxxxxxxxxxxxxx\=\kill
\> \textcolor{sssec}{\underline{Code}} \> \textcolor{sssec}{\underline{Output}} \> \textcolor{sssec}{\underline{Code}} \> \textcolor{sssec}{\underline{Output}} \\

\> {\small\cmd{myeqn\{}} \>\> {\small\cmd{myalign*\{}} \\
\> {\small\texttt{E = m \tb times c\textasciicircum2}}  \> $E = m \times c^2\qquad$ (1.1)  \> {\small\texttt{E\textunderscore i \&= m\textunderscore i \tb times c\textasciicircum2 \tb\tb}}  \> $E_i = m_i \times c^2\qquad$\\
\> {\small\texttt{\cmd{label\{eq11\}}}} \>\> {\small\texttt{y \&= Ax + B}} \> $\:\:\,y = Ax + B$\\
\> {\small\texttt{\}}} \>\> {\small\texttt{\}}}
\end{tabbing}

As is known, the direct versions of the shortcuts will produce the equation number while the * versions will not. A similar coding applies to all the aforementioned defined shortcuts, and the produced output will fit well with proper vertical spacing with text around them. For other specialized equation environments, other requirements, or when things do not fit well, the manual method of issuing the appropriate $-$ve valued arguments with \cmd{\ix{vspace}\{\ldots\}} command can be followed.  
 

%--------------------------------------------------------------
%--------------------------------------------------------------
\subsection{References/bibliography}\index{bibliography}
The two most common bibliography management systems (BMS) are \ixm{bibliography}{BibLaTeX} and \ixm{bibliography}{BibTeX}; the former being modern and highly versatile, and the latter being simpler. BibLaTeX is recommended as the BMS of choice because of its direct usage, versatility, and future-proof capabilities. Reference or bibliography chapter or section can be combined into a \ixm{bibliography}{stand-alone chapter} (whole), or the reference listing can be included in all individual chapters. The bibliography listing in individual chapters, sometimes desired by the user, can be easily coded using the advanced Bib\LaTeX, which is also coded in the class. Both systems (individual or whole) use the same reference data in the form of \texttt{*.bib} file. The authors recommend using any of the systems by appropriately employing the system-specific commands. \index{bibliography!individual chapter} \index{bibliography!whole} Various details of handling bibliography is presented in Sec.~\ref{refhandling} and for bibliography compilation issues see Sec.~\ref{bibissue}. 

%--------------------------------------------------------------
\subsubsection{Cite while you write (CWYW) using natbib}\label{cwyw}\index{bibliography!natbib}
The \texttt{\ix{natbib}} package for bibliography management is widely used and very stable, and follows the CWYW paradigm. The package produces both author-year and numerical citations. The commands like \cmd{citep\{\ldots\}} citation in parentheses and \cmd{citet\{\ldots\}} citation in running text are quite useful in particular. These commands will produce the following outputs, for example: ``(Author et al., 2022)'' and ``Author et al. (2022) found \ldots''. Different reference listing styles can be loaded both for Bib\LaTeX{} and \bt BMS. 


%--------------------------------------------------------------
\subsection{Appendix}

\subsubsection{Single and multiple named appendices}
If the disquisition includes a single \ix{appendix} or multiple named appendices, one of two commands must be used to produce them. If the dissertation has only one appendix, use the \cmd{appendix} command to begin it. This command generates an unlettered APPENDIX chapter that can have sections, subsections, and so on, as well as tables, figures, and other elements.

If multiple named appendices are necessary, use the \cmd{\ix{namedappendices}\{\emph{\ldots}\}\{\emph{\ldots}\}} that can also contain other elements. Following are two examples of the named appendices:
\ccmd{
\cmd{namedappendices\}\{A\}\{First named appendix title here\}}\\
\cmd{namedappendices\}\{B\}\{Second named appendix title here\}}
}

These appendix commands are optional, but are required if the disquisition includes an appendix. The appendix must follow the unnumbered REFERENCES chapter. NDSU's guidelines on appendices only allow named appendices with letters (e.g., ``APPENDIX A'', ``APPENDIX B''), while numerical or other styles (``APPENDIX 1'', ``APPENDIX 2'', ``APPENDIX I'', ``APPENDIX II'', and so forth) are not accepted.

It is necessary to generate the listing of appendices in the TOC up to the subsection level (A.1.1), similar to the regular chapters. To achieve this, the necessary codes were included in the class.

%--------------------------------------------------------------
\subsubsection{Appendix basic figures and tables}\label{apfigtab}
If the \ix{appendix} contains figures or tables, use the \texttt{\ixm{appendix}{appendixfigure}} and \texttt{\ixm{appendix}{appendixtable}} environments to generate them. These special environments have separate counters for appendices and generate LOAT and LOAF entries correctly after the TOC. The usual \texttt{figure} and \texttt{table} environments in appendices will not add the entries in LOAT and LOAF as expected (will only go to LOT and LOF). The same rules for centering, captions, and labels used in normal \texttt{figure} and \texttt{table} environments apply to \texttt{appendixfigure} and \texttt{appendixtable} environments. 

As the coding for figures is simple, shortcuts for \texttt{appendixfigure} are possible with basic commands, while shortcuts for \texttt{appendixtable} are not and need to be done in the usual manner. Similar to figures handled in the regular chapters (Sec.~\ref{figs}), single command shortcuts for appendix figures and appendix landscape figures were defined. Following are the examples of figure shortcuts for appendix regular and landscape figures: \index{appendix!myfigap} \index{appendix!myfigapls}

\begin{flushleft}
\hspace{-0.2cm}
\begin{minipage}{0ex}
\begin{verbatim}
 \myfigap{H}{0.6}{image_ap1.jpg}{Caption for this appendix regular figure}{fig:ap1}
 \myfigap[12mm]{H}{0.6}{image_ap2.jpg}{Appendix landscape figure caption}{fig:ap2}

 \myfigapls{p}{1.32}{image_ap3.pdf}{Caption for this appendix landscape figure}{f:ap3ls}
 \myfigapls[-1.5ex]{p}{1.33}{image_ap4.pdf}{Appendix landscape figure caption}{f:ap4ls}
\end{verbatim}
\end{minipage}
\end{flushleft}


%--------------------------------------------------------------
\subsubsection{Appendix advanced elements---longtables, listings, schemes, and multipage figures}\label{apfigtab}
Handling of advanced elements in appendices (e.g., longtables, code listings, schemes, single-page subfigures, and multipage subfigures) needs additional consideration as these are handled only through \texttt{appendixfigure} and \texttt{appendixtable} environments---as of now in the class. This means listings, schemes, and subfigure types will be considered as entries of \texttt{appendixfigure} and longtables as \texttt{appendixtable}, and these elements will be listed in the LOAF and LOAT only, respectively. As separate entries will make it bloated, the above strategy will be a compact solution. 

The method these elements are addressed in appendices in the class is: (i) code the respective elements using their natural environment (e.g., \texttt{scheme, subfigures, longtable}; no class defined shortcuts) in the appendices without their caption and label---which will produce the elements in the usual manner; (ii) code a dummy \texttt{appendixfigure} or \texttt{appendixtable} with ``only caption and label'' corresponding the element source ---which will create appropriate caption nead the elements and an entry in LOAF or LOAT. The label can be cross-referenced as usual and will have the reference number corresponding to the named appendix. This workaround is versatile and could include future elements to be dealt with in the appendices. The shortcuts, expanded as ``my-figure-appendix-caption'' (\cmd{myfigapcap}) and ``my-table-appendix-caption''  (\cmd{mytabapcap}), produced for the dummy element appendix figure and table captions (1 optional spacing and 2 mandatory caption and label arguments) are shown below. Refer to the ``\texttt{NDSU-Thesis-Extended.tex}'' (URL: \url{https://github.com/CIgathi/NDSU-Thesis-Class}) for an example code. \index{appendix!myfigapcap} \index{appendix!mytabapcap}

\begin{flushleft}
\hspace{-0.2cm}
\begin{minipage}{0ex}
\begin{verbatim}
 \myfigapcap{Caption for this appendix listings, schemes, or multipage figures}{fig:ap1a}
 \myfigapcap[6pt]{Similar appendix figure caption with optional spacing argument}{f:ap2a}

 \mytabapcap{Caption for this appendix longtables}{tab:ap1a}
 \mytabapcap[-1cm]{Similar appendix table caption with optional spacing argument}{t:ap2a}
\end{verbatim}
\end{minipage}
\end{flushleft}

 
%--------------------------------------------------------------
\subsubsection{Closing appendices and creating TOC, LOAT, and LOAF}\label{apfigtab}
In the class, the creation of \ixm{appendix}{LOAT} and \ixm{appendix}{LOAF} as well as their TOC entries requires special consideration. When the last appendix has at least one table and one figure, the TOC, LOAT, and LOAF will be automatically generated without intervention. However, when the last appendix does not have at least a table, or figure, or both (even though the previous appendices had them), the corresponding LOAT or LOAF and relevant TOC entries will not appear. The solution for the automatic creation of these items, irrespective of the contents of the last appendix is issuing the relevant commands such as \cmd{\ix{closeappendixtables}}, \cmd{\ix{closeappendixfigures}}, or \cmd{\ix{closeappendices}} in the overall code as the last command before \cmd{end\{document\}} and after the last appendix code. Several scenarios occur due to the presence or absence of appendix tables and figures in the last or previous appendices, and the last command (or) none to be used are presented in Table~\ref{appclose} for proper generation of LOAT and LOAF. 

\begin{table}[h!]
\centering
\caption{Appendix closing for LOAT and LOAF generation---last command before \cmd{end\{document\}}}
\vspace{-1.5ex}
\begin{tabular}{l l l p{2.05in}}
\toprule
Previous appendix(s) & Last appendix & Command & Comment\\
has/have & has & to be used\\
\midrule
No Table \& Figure  & No Table \& Figure  & None & LOAT and LOAF not generated\\
No Table \& Figure  & Only Table   & None & Only LOAT auto generated\\
No Table \& Figure  & Only Figure   & None & Only LOAF auto generated\\
No Table \& Figure  & Table \& Figure  & None & LOAT and LOAF auto generated\\
\midrule
Only Table   & Only Table   & None & Only LOAT auto generated\\
Only Table   & Only Figure   & {\small\cmd{\ix{closeappendixtables}}} & LOAT and LOAF generated\\
Only Table   & No Table or Figure   & {\small\cmd{\ix{closeappendixtables}}} & Only LOAT generated\\
\midrule
Only Figure   & Only Table   & {\small\cmd{closeappendixfigures}} & LOAT and LOAF generated\\
Only Figure   & Only Figure   & None & Only LOAF auto generated\\
Only Figure   & No Table or Figure   & {\small\cmd{\ix{closeappendixfigures}}} & Only LOAF generated\\
\midrule
Table \& Figure  & Only Table*  & {\small\cmd{closeappendixfigures}} & LOAT and LOAF generated\\
Table \& Figure  & Only Figure*  & {\small\cmd{closeappendixtables}} & LOAT and LOAF generated\\
Table \& Figure  & No Table or Figure   & {\small\cmd{closeappendices}}  & LOAT and LOAF generated\\
\bottomrule
\multicolumn{3}{l}{\small * The command  \cmd{closeappendices} can also be used instead} 
\end{tabular}
\label{appclose}
\end{table}

The aim is to generate the LOAT or LOAF or both as coded in the appendices, avoiding blank lists with only headers or no lists altogether. The other items of the TOC, namely, LOT, LOF, LOSH, LOA, and LOS, are automatically generated based on the presence or absence of these floats/items coded in the chapters or prefatory material.  

%--------------------------------------------------------------
%--------------------------------------------------------------
\section{Additional Information \textrm{I}---Special Commands}

%-------------------------------------------
\subsection{Chapter styles}\index{chapter styles}
Two styles, namely,  regular- and paper-styled chapters, are generally followed. The regular is a traditional style where the whole thesis/dissertation is considered as a single document where individual chapters exclusively deal with aspects like introduction, literature review, methods, results, discussion or results and discussion, references, and appendices reflecting all studies carried out in the research on these individual chapters. Even though this style produces a consolidated document and is solid in its own merit, which ties all research aspects of the study together in corresponding chapters, a good deal of rewriting will be necessary from the authors if they want to publish the contents as individual peer-reviewed journal articles.

The paper-styled chapters are stand-alone chapters complete with all sections (abstract, introduction, literature review, \ldots, references) and are the modern trend. In this style, some amount of repetition among chapters is unavoidable (especially in methods, analysis, and references). However, as the chapters are already in paper-style, it is very easy to format them to suit the requirements of any peer-reviewed journal for submission. It is also possible to have individual chapter references (Bib\LaTeX) or a combined reference chapter (both Bib\TeX\ and Bib\LaTeX). With \LaTeX\ it is easy to create stand-alone papers with references for submission from a paper-styled disquisition with a combined reference chapter.  As outlined earlier,  the commands that start these chapters are 
\cmd{myheading\{\ldots\}} or \cmd{mypaperheading\{2 args\}} (Sec.\,\ref{heading}).  It is a good idea to consult the advisor before committing to these styles,  for they are different, and substantial rewriting is involved to switch back and forth. 

%-------------------------------------------
\subsection{Advanced options in documentclass}

%--------------------------------------------------------------
\subsubsection{Font size}
The general \ix{font sizes} used with the thesis are 10, 11, and 12 points, and they vary with the selected font. The available options (any of these used) are:
\ccmd{\cmd{documentclass[10pt (or) 11pt (or) 12pt]\{ndsu-thesis-2022\}}}
The default was set as {\tt{12pt}}.

%--------------------------------------------------------------
\subsubsection{Auto-numbered, chapter-numbered, and unnumbered styles}
The three possible NDSU thesis styles with options included are: (i) Auto-numbered [default option]---where chapters, sections, subsections, and so on will be numbered; (ii) Chapter-numbered [\ix{chapternumber}]---where only chapters are numbered, while sections, subsections, and so on will not be numbered; and (iii)~Unnumbered [\ix{nonumber}]---where all headings such as chapters, sections, subsections, and so on will not be numbered.

As the default is the numbered style, the chapter-numbered and unnumbered styles were produced by the ``chapternumber'' and ``nonumber'' options, respectively as: \ccmd{\cmd{\ix{documentclass}[chapternumber (or) nonumber]\{ndsu-thesis-2022\}}}. The default was the ``Auto-numbered'' style. These options will have their specific effect on the numbering scheme of the tables and figures.

%--------------------------------------------------------------
\subsubsection{Paragraph text justification}
Based on their preference, students can follow \ix{fully-justified} (with hyphenated words and word wrapping) or unjustified (no word breaking but right margin ragged, aka left-justified). As the default is justified, the \ix{left-justified} passages were produced by the \texttt{[\ix{nojustify}]} option. For justified style, nothing needs to be specified. NDSU approves both styles.

%--------------------------------------------------------------
\subsubsection{Draft and display \ix{document frames}}
You can use the \texttt{[\ix{draft}]} option to place the disquisition into draft mode. In this mode, \ix{margin overflows} are marked with a heavy black box to draw your attention to them; additionally, images are replaced by a \ix{placeholder} (Fig.~\ref{fig:frame}a). If you import other packages in your disquisition, they may also change their behavior when in draft mode.

\begin{figure}[h!]
\centering
\vspace{-0.8ex}
\includegraphics[width=\textwidth]{fig-draftmargin-grids.pdf}
 \caption{Use of (a) \texttt{draft} and \texttt{showframe} options in \texttt{documentclass} producing image placeholder for quicker processing, document frames, and margin overflows, and (b) use of \texttt{showgrid} option displaying grids of \qty{0.1}{in} squares spacing to help visualize the alignment (vertical and horizontal) concerns of elements.}
\label{fig:frame}
\end{figure}

\indent The \texttt{[\ix{showframe}]} option (based on \texttt{\ix{geometry}} package) produces a frame around the text area, which can be used to check how the text aligns with the margins (left, right, top, and bottom; see figure above). The illustration alongside displays the result of these options, showing the overflowing text, bottom \ix{margin frame}, right margin frame, margin notes frames, and the \ix{overflow} heavy black box. The default behavior is that these options were inactive.

%--------------------------------------------------------------
\subsubsection{Grids display}\index{grids}
While alignment and spacing of elements will be automatically handled by \LaTeX, the NDSU's guidelines deviate a bit from the \LaTeX\/ engine, especially texts around non-textual elements, viz. equations, tables, and figures. Although the normal behavior of \LaTeX{} introduces a little extra space around non-textual elements, NDSU's guideline dictates a consistent double-spacing that is followed in the regular text should be around the non-textual elements. This additional spacing can be corrected by using a $-$ve \cmd{\ix{vspace}\{\ldots\}} command before and after the elements as required (see Sec. \ref{spacearound} for details). 

To visualize better the spacing around textual and non-textual elements, a documentclass \texttt{[showgrid]} option was made available (Fig.~\ref{fig:frame}b). The grid was displayed underneath the elements covering the body (\texttt{textwidth} $\times$ \texttt{text height}), and the lines were spaced at \qty{0.1}{in} on both vertical and horizontal directions, and the whole page is rendered on a light gray background. It can be seen that the basic text line vertical spacing is about \qty{0.4}{in} (4 grid lines between the baselines of consecutive lines of text; Fig.~\ref{fig:frame}b). This vertical spacing (\qty{0.4}{in}) should be carried throughout the document with some deviation among captions of tables and figures, table data rows, and program listings. After fixing the spacing concerns, using \cmd{\ix{vspace}\{\ldots\}} commands, the \texttt{showgrid} option should be removed for the final output. By default, this option is inactive. 

%--------------------------------------------------------------
\subsubsection{Fonts}\index{font}
The following \ix{font options} \texttt{[\ixm{font}{bookman}, \ixm{font}{charter}, \ixm{font}{gentium}, \ixm{font}{kpfonts}, \ixm{font}{libertine}, \ixm{font}{mathdesign}, \ixm{font}{mathptmx}, \ixm{font}{mlmodern}, \ixm{font}{newcent}, \ixm{font}{newpx}, \ixm{font}{newtx}, \ixm{font}{palatino}, \ixm{font}{tgtermes}, \ixm{font}{times}, \ixm{font}{tgbonum}, \ixm{font}{tgpagella}, \ixm{font}{tgschola}, \\\ixm{font}{utopia}, \ixm{font}{clearsans}, \ixm{font}{cmbright}, \ixm{font}{firasans}, \ixm{font}{helvet}, \ixm{font}{kurier},\ixm{font}{lxfonts}, \ixm{font}{sansmathfonts}]} (both serif and san serif fonts) are loaded and compatible with the class, and anyone can be used. It is also possible to use several other fonts from ``The \LaTeX{} Font Catalog'' web resource (\url{https://tug.org/FontCatalogue/}) and add the code given in the ``Usage'' section in the preamble of the main document \texttt{*.tex} file. The default was \LaTeX\ computer modern font. Users are urged to check the NDSU-approved fonts and select those that resemble them and use them with appropriate font sizes. 

%-------------------------------------------
\subsection{Line numbers for the whole document}
Sometimes using line numbers will be helpful while communicating with the advisor or others, where specific locations of the document can be pointed to. Line numbers are generated using the package \texttt{lineno}, which is coded into the class, by the following command:
\ccmd{\cmd{\ix{linenumbers}}}
This command can be issued at the beginning or at any point, and numbers will appear in the left margin after the command. Of course, this command should be removed or commented on while finalizing the thesis.

%-------------------------------------------
\subsection{Whole document text spacing}
NDSU mandates double-spacing for the body paragraphs' text. A default double-spacing setting in MS Word produces 23 lines per page while \LaTeX\ \cmd{\ix{doublespacing}} produces 27 lines. Any of these lines per page defaults in the respective systems is acceptable. To recreate the line spacing of 23 lines per page in \LaTeX\ was reproduced through \cmd{renewcommand\tb \ix{myspacing}\{1.9\}} in the preamble, which was also set as the default in the class, and this spacing is automatically applied to the whole document.  

Other values of \cmd{myspacing} will produce other spacings, but should be consulted before using them. While developing the document and working with drafts, it will be efficient to reduce the \cmd{myspacing} value to less than 1.0 (e.g., 0.75) to generate a compact version for reviewing and draft printing, and can be restored to 1.9 for final disquisition in the correct line spacing. 

%--------------------------------------------------------------
\subsection{Tables: Advanced commands}
\subsubsection{Table row spacing and fonts}
The table contents row spacing can be adjusted, if desired, using \cmd{renewcommand}\cmd{\ixm{table}{arraystretch}\}\{decimal\}} before \texttt{tabular} and \cmd{SetTblrInner\{rowsep=length\}} before \texttt{tbrl} environment inside the table environment. For example, a value of 1.75 for the arraystretch will be similar to the double line spacing, and without this command, the row spacing will be single line spacing. Table footnotes can be added through \texttt{\ixm{table}{tablenotes}} environment placed inside \texttt{table} environment after the \texttt{tabular} and \texttt{resize} blocks. The NDSU template font footnote is the regular font; however, the font size can be altered by selecting the standard sizes (e.g., \cmd{footnotesize}, \cmd{small}) within the \texttt{tablenotes} environment to match the content of the table.

%--------------------------------------------------------------
\subsubsection{Landscape tables}
When a table has more columns of information, the most common solution is the landscape orientation which is achieved through \texttt{\ixm{table}{landscape}} environment by enclosing the \texttt{table} codes (which may contain other elements) inside \texttt{landscape} environment block (between \cmd{begin\{landscape\}} and \cmd{end\{landscape\}}). With landscape, usually the placement option will be [p] and the whole width should be set around 1.32 times the \cmd{\ix{columnwidth}}, or adjusted suitably to leave acceptable margins all around.

%--------------------------------------------------------------
\subsubsection{Resizing tables for full-width} \index{table!resize}\index{table!full-width}
Sometimes, while fitting more content into a table, the table extends beyond the allowed margins. Therefore, for the best control of tables, especially with more columns, a combination of \cmd{resizebox} (resizing the entire table - mostly for scaling down) and \cmd{tabcolsep} (maintaining the column separation space) works the best. Thus, the command \cmd{\ixm{table}{resizebox}\{\cmd{\ixm{table}{columnwidth}}\}}\{!\}\{ \texttt{\ldots table codes \ldots} \} makes the table span the entire text width of the page. This will expand or shrink the contents of the table to fit the entire width. It should be fine with the fonts shrinking to fit the width, but it will not be when the fonts enlarge (especially when the table is small and has only a few columns). In such situations, the space between the columns can be adjusted using the \cmd{\ixm{table}{tabcolsep}\{\ldots\}} command, where increased spacing reduces the font size and \emph{vice versa}. Thus, by using these commands (including \texttt{tblr}) in combination, the tables and the font size can be scaled down to fit the page with the proper font size.

%--------------------------------------------------------------
\subsubsection{Long tables}
Sometimes, there will be a need to create long tables (multiple pages spanning tables) in the landscape orientation to accommodate several columns that will not fit in regular paper orientation. This landscape long table can be logically obtained by enclosing the long table codes within \texttt{landscape} environment block as described earlier. It should be noted that, unlike regular tables, long table source code involves several components (main caption, running header [abbreviated caption], running footer [usually the word ``continued\ldots''], and main footnote). Refer to the ``\texttt{NDSU-Thesis-Extended.tex}'' (URL: \url{https://github.com/CIgathi/NDSU-Thesis-Class}) for an example code. 

%--------------------------------------------------------------
\subsubsection{Landscape long tables}\index{table!landscape longtable}
Combining the \texttt{landscape} and \texttt{longtable} environments, the landscape long tables can be created logically. Refer to the ``\texttt{NDSU-Thesis-Extended.tex}''  (URL: \url{https://github.com/CIgathi/NDSU-Thesis-Class}) for an example. 

%--------------------------------------------------------------
\subsection{Figures: Advanced commands}

%-------------------------------------------
\subsubsection{Figures and schemes in separate folder}
Several images (figures, schemes, graphs, drawings, and pictures) were used while developing a thesis or paper. It will be convenient to store all these images in a subfolder to reduce the clutter. The following command should be issued in the preamble, indicating the name of the subfolder (e.g., \texttt{figures}) relative to the main ``\texttt{tex}'' file as:
\ccmd{\cmd{\ix{graphicspath}\{\{./figures/\}\}}}
The type of image files applicable are: \texttt{jpg, pdf, png,} and  \texttt{eps}. It is also possible to give an absolute path to the images folder in the above command.

%--------------------------------------------------------------
\subsubsection{Flowchart - \ix{tikz} package}
Flowcharts, schemes, geometrical diagrams, circuit diagrams, and data visualization graphs are common in technical writing. These elements can be created elsewhere and included in the dissertation as an image, or high-quality (vector graphics) can be created using code directly. An example of a schematic \ix{flowchart} created through Ti\emph{k}Z code is shown below:  \index{tikzpicture}

\usetikzlibrary{shapes,arrows,shadows}

\begin{figure}[h!]
\setcounter{figure}{0}% to make Schematic 1 - no newfloat coded
\renewcommand{\figurename}{Schematic}% for scheme
\centering
\begin{tikzpicture}[node distance = 3.5cm, every node/.style={scale=0.85}]
\tikzstyle{block} = [rectangle, drop shadow, draw, fill=green!30, text width=7em, text centered, rounded corners]
\tikzstyle{line} = [draw, -latex']
\tikzstyle{cloud} = [draw, drop shadow, ellipse, fill=pink, minimum height=2.6em]

    \node [cloud] (start) {\texttt{Start}};
    \node [block, right of=start, node distance = 3cm] (read){Read inputs};
    \node [block,  fill=blue!15, right of=read] (pros) {Process data \\ (Data cleaning and grouping)};
    \node [block, fill=blue!15, right of=pros] (ana) {Analyze results \\ (Fit model and interpret)};
    \node [block, text width=11em, node distance = 4.2cm, right of=ana] (rep) {Generate reports \\ (For general and technical consumptions with images and tables)};
    \node [cloud, right of=rep,node distance = 3.5cm] (stop) {\texttt{End}};
    \path [line, thick] (start) -- (read);
    \path [line, thick] (read) -- (pros);
    \path [line, thick] (pros) -- (ana);
    \path [line, thick] (ana) -- (rep);
    \path [line, thick] (rep) -- (stop);
\end{tikzpicture}
\caption{A high-quality schematic flowchart created using the Ti\emph{k}Z package.}
\label{fig:fc}
\end{figure}

The Ti\emph{k}Z package, based on \TeX\ is an excellent and elaborate package (manual having $>1300$ pages) that can be used for creating high-quality graphics that serve the needs of any technical documentation (Example Fig.~\ref{fig:fc}). Going through the manual of Ti\emph{k}Z and the gallery will give information on the package capabilities and how they can be used in the dissertation. 
\setcounter{figure}{2}

%--------------------------------------------------------------
\subsubsection{Subfigures}
Multiple figures (\ix{subfigures}) under a common caption can be handled through \texttt{subfig} package. The subfigures can be individually sized, captioned, labeled, and referenced. The sub-caption numbering is ``alphabetic'' by default (holds 26---and for more subfigures, other options are available) and will be automatically generated. The number of images that occupy a single row can be readily coded with commands, such as \cmd{\ix{subfloat}\{\ldots\}}, \cmd{hspace\{\ldots\}}, and newline (\tb\tb). Refer to the accompanied ``NDSU-Thesis-Extended'' (URL: \url{https://github.com/CIgathi/NDSU-Thesis-Class}) document for instructions and examples.

%--------------------------------------------------------------
\subsubsection{Landscape subfigures}\index{figure!landscape subfigures}
Similar to landscape long tables, subfigures that span multiple pages can be enclosed in \texttt{landscape} environment block to produce landscape multiple pages of subfigures with the same figure number. Refer to the ``\texttt{NDSU-Thesis-Extended.tex}'' (URL: \url{https://github.com/CIgathi/NDSU-Thesis-Class})for example code.   

%--------------------------------------------------------------
\subsubsection{Multiple page subfigures}\index{figure!subfigures multipage}
Sometimes several subfigures running through multiple pages need to be used in the thesis. These are similar to long tables that span several pages.  General captions of the set of subfigures fitting in a page are coded with the regular \cmd{caption}\{\ldots\} command, which can be individually controlled. These captions may be the same or abbreviated. Subfigures and their sub-captions were created through \cmd{subfloat[\ldots]\{\ldots\}} command of the \texttt{subfig} package. The optional argument of \cmd{subfloat} is the individual subfigure caption, and the regular argument is the \cmd{includegraphics} command with its usual optional argument. The \cmd{ContinuedFloat} with another \texttt{figure} environment will carry the numbering forward. When the number of subfigures exceeds the number of alphabets (26), the numbering system should be switched to numeric. Refer to the ``\texttt{NDSU-Thesis-Extended.tex}'' (URL: \url{https://github.com/CIgathi/NDSU-Thesis-Class}) for example code showing multiple page subfigures.

%-------------------------------------------
\subsection{Spacing adjustment around non-textual elements}\label{spacearound}
Usually, the \ix{spacing} around the non-textual elements produced by \LaTeX\ will be good and based on typography principles. The environments that create these elements (e.g., tables, figures, schemes, equations) automatically supply an additional space to set the elements apart from the regular text, and this is the expected and correct behavior. However, sometimes additional space will appear above or below these elements, which may be the result of fitting the elements with respect to others in the whole chapter. The spacing around the non-textual elements can be altered by one or any combination of the following to produce a consistent spacing around the non-textual elements---which is an NDSU thesis requirement:
\begin{itemize}[itemsep=0em, leftmargin=*]
\item
An additional document named ``ndsu-thesis-vertical-spacing'' that addresses this is available in the package (URL: \url{https://github.com/CIgathi/NDSU-Thesis-Class})\index{vertical spacing}. 
\item
The issue becomes apparent when the full page float is taken to the subsequent page and ``vertically'' centered. This breaks the rule of a ``1-inch'' top margin. The solution in simpler terms for this is: (i) Give a \cmd{newpage} command before the float environment, (ii) use placement specifier \texttt{[h!]} or \texttt{[H]} not \texttt{[p]}, and (iii) the blank space before the float is okay as this larger float could not have been accommodated there in the first place. 
\item
The blank line coded, usually left between paragraphs, might create additional space before the element (e.g.,\texttt{equation}, \texttt{align}), and that can be removed to reduce the space above the element. For equations, the defined shortcuts can be used to produce the correct vertical spacing (see Sec.~\ref{eqnshortcuts}). 
\item
Proper use of the vertical spacing \cmd{\ix{vspace}\{\ldots\}} command with \ix{negative spacing} (e.g., \cmd{vspace\{-3ex\}}) can able to correct the blank space above the element. This can also be used when a blank line was issued to separate the regular text from the element. Positive vertical space can also be issued as needed.

\item
In figures and schemes, the space above the caption (the space between the bottom of the image and the top of the caption) can be controlled by using the optional argument of the \texttt{myfig, myfigls, mysch, myschls, myfigap} and \texttt{myfigapls} commands (see Sec.~\ref{figureshortcut}). This optional argument was specifically developed to address this caption placement issue. This may be required only for necessary adjustments, as the default (without option) will work well in most cases.
\end{itemize}

%--------------------------------------------------------------
\subsection{Reference handling using \ix{Bib\LaTeX\ }}\label{refhandling}
The Bib\LaTeX\ package provides advanced bibliographic facilities for use with LaTeX. A good working knowledge of LaTeX should be sufficient to design new bibliography and citation styles using this system. The  \blt works with the backend (program) ``\ixm{Bib\LaTeX\ }{biber}'', which is used to process the bibliography data files and then performs all sorting, label generation, and many more operations. This package also supports subdivided bibliographies, multiple bibliographies within one document, customizable sorting, \ixm{Bib\LaTeX\ }{multiple bibliographies} with different sorting, customizable labels, and bibliographies may be subdivided into parts and/or segmented by topics. Users are urged to refer to the package documentation for various features (\url{https://ctan.org/pkg/biblatex?lang=en}).

%--------------------------------------------------------------
\subsubsection{Commands, cite, bibliography generation, and files handling}
\label{bibgen}

The basic commands that invoke the \blt system, which have to be issued in the preamble, are:
\ccmd{
\cmd{usepackage[style=apa,natbib=true,backend=biber]\{biblatex\}}\\
\cmd{\ixm{Bib\LaTeX\ }{addbibresource}\{\textit{name-of-bib-file.bib}\}}
}

In the above \blt command's option, the bibliography information processing program ``biber'' was used as a backend program. The options also load ``apa'' style and ``natbib'' handling (allowing the \cmd{citep\{\ldots\}} and \cmd{citet\{\ldots\}} commands in  \blt ) as an example. The documentation and other resources (\url{http://tug.ctan.org/info/biblatex-cheatsheet/biblatex-cheatsheet.pdf}) may be referred to for common \ixm{Bib\LaTeX\ }{options} and details of the package. The compatible styles used with \blt are: \texttt{numeric, numeric-comp, alphabetic, authoryear, authoryear-icomp, authortitle, verbose, reading, \\draft, apa, chem-acs, chem-angew, chem-biochem, chem-rsc, ieee, mla, musuos, nature, nejm, phys, science,}, \texttt{oscola}, and so on (new styles are made available with every update). Users can use an appropriate style to match their specialization style guide.

\vspace{1ex}
\textcolor{sssec}{\textbf{Note}}: Only with numerical style bibliography, such as \texttt{ieee}, \texttt{nature}, \texttt{numeric}, \texttt{numeric-comp}, \texttt{\ldots}, to have the numbers in the reference listing left-aligned (especially $> 10$ items), it is best to copy this code and paste it after the \cmd{addbibresource\{\ldots\}} command in the thesis source (*.tex) file (around line 15).  

\vspace{10ex}
\begin{verbatim}
\makeatletter % Make @ available for internal macros
    \defbibenvironment{bibliography}
    {\list
        {\printtext[labelnumberwidth]{%
        \printfield{labelprefix}%
        \printfield{labelnumber}}
        }%
        {\setlength{\labelwidth}{\labelnumberwidth}%
        \setlength{\leftmargin}{\labelwidth}%
        \setlength{\labelsep}{\biblabelsep}%
        \addtolength{\leftmargin}{\labelsep}%
        \setlength{\itemsep}{\bibitemsep}%
        \setlength{\parsep}{\bibparsep}}%
        \renewcommand*{\makelabel}[1]{##1}
        }
    {\endlist}
    {\item}
\makeatother 
\end{verbatim}
\vspace{1ex}


With the package and bib file(s) loaded and processed, the reference listing can be generated anywhere in the document by issuing:
\ccmd{
\cmd{\ixm{Bib\LaTeX\ }{printbibliography}[heading=bibintoc,title=REFERENCES]}
}

The options ``\texttt{heading=bibintoc}'' make an unnumbered chapter and include the heading in the TOC, and ``\texttt{title=REFERENCES}'' changes the default title from BIBLIOGRAPHY to REFERENCES. \\ The\cmd{printbibliography} command, when issued at the end of the chapters, will create a ``combined'' \mbox{REFERENCES} chapter. 

%--------------------------------------------------------------
\subsubsection{Automatic individual (multiple) and whole document reference listing using \blt}
Sometimes it is desired to have a bibliography listing in every chapter, as the last unnumbered section, especially with the paper-style chapters. These individual chapter reference listings can be invoked by the option \texttt{[chapterrefs]} in the documentclass. Individual chapters' bibliography can be easily processed using \blt rather than \ix{\bt} through \texttt{refsection} environment. 

New commands such as \cmd{\ix{checkBeginRefsection}} and \cmd{\ix{checkEndRefsection}} can be considered as an environment that immediately follows each chapter title and encloses the chapter contents (similar to \texttt{\ix{refsection}} environment)---as shown subsequently. At the end, usually before appendices, the \\\cmd{\ix{checkMakeCombinedReferences}} command was once issued. 

\ccmd{
\cmd{mypaperheading\{\ldots\}\{\ldots\}} \texttt{\% chapter n} \\[1ex]
\cmd{checkBeginRefsection}\\[1ex]
\textcolor{gray}{\hspace{0.2in}\ldots Chapter's text starting with abstract, sections/subsections, and so on,} \\
\textcolor{gray}{\hspace{0.2in}with citations using \cmd{citep\{\ldots\}} and \cmd{citet\{\ldots\}} or  other citation commands \ldots} \\[1ex]
\cmd{checkEndRefsection} \\[1ex]
\cmd{mypaperheading\{\ldots\}\{\ldots\}} \texttt{\%chapter n+1} \\[1ex]
\cmd{checkBeginRefsection}\\[1ex]
\textcolor{gray}{\hspace{0.2in}\ldots Chapter's text starting with abstract, sections/subsections, and so on,} \\
\textcolor{gray}{\hspace{0.2in}with citations using \cmd{citep\{\ldots\}} and \cmd{citet\{\ldots\}} or  other citation commands \ldots} \\[1ex]
\cmd{checkEndRefsection}\\[1ex]
\textcolor{gray}{\ldots Other chapters' codes follow a similar arrangement} 
 \\[1ex]
}

\ccmd{
\cmd{checkMakeCombinedReferences} \texttt{\% Given at the end of chapters}\\[1ex]
\textcolor{gray}{\ldots Appendices codes \ldots}
}

A combination of these commands---automatically checks the desired type of reference listing---and produces the individual chapter reference listings (with \texttt{[chapterrefs]} documentclass option) or the whole document unnumbered reference chapter (without the aforementioned option - default). Following these automatic commands of \blt will be the most efficient way of handling the bibliography. Examples of the usage of these commands will be seen in the included ``\texttt{Sample-thesis-IncludeOnly}'' folder files. Needless to mention, these are convenient commands developed for the class, and the desired behavior can also be achieved by usual direct commands. 

%--------------------------------------------------------------
\subsection{Reference handling using \ix{\bt}}

%--------------------------------------------------------------
\subsubsection{BibTeX}\label{cwyw}\index{bibliography!natbib}

The compatible styles (\texttt{*.bst}) with \texttt{natbib} and NDSU class that work with standard \LaTeX{} installation are:  \texttt{abbrvnat, agsm, agu, apalike, apalike2, authordate1, authordate3, cell, chicago, chicagoa, dcu, dinat, IEEEtran (family;  numerical styles), kluwer, plainnat, rusnat, unsrtnat}, and so on\index{natbib styles} (new styles are made available with every update). For other styles, users can able to download the specific style (\texttt{*.bst}) files and have them in the local folder. As \texttt{natbib} is an optional bibliography system, it was not coded in the class, and to use \texttt{natbib}, the following code should be in the preamble:
\ccmd{
\cmd{usepackage[sort\&compress]\{natbib\}}\\
\cmd{\ixm{bibliography}{citestyle}\{arms\} \% agms, agu, arms, egu, cospar, dcu, kluwer, plain, nature}
}

It is convenient to load the \texttt{natbib} package with minimal options, as shown above, and choose predefined \cmd{citestyle\{option\}} options producing several styles defined in the package. 

%--------------------------------------------------------------
\subsubsection{Bibliography generation and files handling}\label{makebib}
The two basic commands that are required to implement \bt system are:
\ccmd{
\cmd{bibliographystyle\{\textit{style}\} \% See list of styles (Sec.~\hspace{-1ex}11.1.1)}\\
\cmd{bibliography\{\textit{name-of-bib-file}\}}
}

However, a single new command ``\cmd{makebib}'' (direct shortcut with no arguments) was coded, replacing the above commands, setting up the bibliography style and *.bib file arguments as: 
\ccmd{
\cmd{newcommand\cmd{makebib}\{\cmd{biblio}\{\textit{style}\}\{\textit{name-of-bib-file}\}\}}
}\index{makebib}

The above command is conveniently placed at the beginning of the preamble, where the users replace the style and bibliography arguments as inputs. Then, simply issuing the \cmd{makebib} will generate the references listing based on the style and bib file input in the above command. The ``style'' of bibliography (\texttt{*.bst}) entries (typically \texttt{plainnat} or \texttt{apalike}), is controlled by the first argument; the user is referred to the \bt manual for formatting details and other available styles, such as those provided by the peer-reviewed journals related to the specialization. The ``name-of-bib-file'' used in the second argument must be the same as the name of the bibliography (\texttt{*.bib}) file, but with the extension removed. Once correct citation commands (\cmd{\ix{citep}\{\ldots\}}  and/or \cmd{\ix{citet}\{\ldots\}}) are issued following CWYW, the citation with proper reference number or entry will appear in the text and listings in the proper style (based on *.bst) will be generated. The above shortcut generates an unnumbered chapter with the title REFERENCES (accepted by NDSU) and also a corresponding TOC entry.

These commands (or equivalent commands if the user uses a different bibliography management system) are optional but are required if the disquisition includes references. The basic bibliography citation command is \cmd{cite\{\ldots\}}, and that works as well. \index{bibliography!citep} \index{bibliography!citet}


%--------------------------------------------------------------
\subsection{Specialization specific bibliography---Examples of ASABE and IEEE}
The bibliography styles to be followed in the thesis/dissertation will be based on the parent department and the major technical society the department subscribes to, in general. For example, the NDUS's ``Agricultural and Biosystems Engineering'' department's major technical society is the American Society of Agricultural and Biological Engineers (ASABE), and the NDSU's ``Electrical and Computer Engineering'' department's technical society is ``The Institute of Electrical and Electronics Engineers (IEEE).'' The two ways of making the bibliography are to use \blt or \bt. The ASABE style journal references can be reproduced using \cmd{usepackage[style=apa,} \texttt{natbib=true]} \texttt{\{biblatex\}}. 

For \texttt{IEEEtran} styles, with \blt any of the following can be used (with increasing functionality). With \bt \texttt{*.bst} files and \cmd{citestyle\{\ldots\}} commands are available to create the style of reference followed in various departments (see Sec.~\ref{refhandling} for details). 

\vspace{1ex}
$\qquad$\cmd{usepackage[style=ieee,backend=biber]\{biblatex\}}\par
$\qquad$\cmd{usepackage[style=ieee,natbib=true,backend=biber]\{biblatex\}}\par
$\qquad$\cmd{usepackage[style=ieee,natbib=true,citestyle=numeric-comp,backend=biber]\{biblatex\}}\\


%--------------------------------------------------------------
\subsection{Bibliography compilation issues}\index{bibliography!compilation}\index{bibliography!issues}\label{bibissue}
The common issues \textcolor{sssec}{[and proposed solution]} while working with bibliographies, especially in combination with \texttt{natbib} include:
\vspace{-1ex}
\begin{itemize}[itemsep=0em, leftmargin=*]

\item The basic principle usually gets violated is that the \LaTeX{} inputs (\texttt{tex, bib, or others}) require only the ASCII characters. Anything that disregards this principle will have issues during compilation. Presence of extraneous characters (unprintable and formatting) in the \texttt{*.bib} file, which comes while copying a formatted non-ASCII text, stalls the compilation. Command line or terminal commands or online tools can be used to identify the non-ASCII characters (e.g., \url{https://pages.cs.wisc.edu/~markm/ascii.html}, \url{https://onlineasciitools.com/validate-ascii}) \textcolor{sssec}{[remove non-ASCII text from \texttt{*.bib} file]} 

\item \bt contains several types of entries (e.g., article, book, misc, etc) and each has its own ``Required'' and ``Optional'' fields. Not filling the required field will throw an error \textcolor{sssec}{[familiarize with these fields (\url{https://en.wikipedia.org/wiki/BibTeX}) and ensure supplying all the required fields or change to a suitable different entry that is less restrictive (e.g., misc)]}  

\item Another most common issue with bibliography entries is ``duplicate entries.'' These are entries with repeated citation ``keys'' (non-unique) and stop the compilation with a clear error message \textcolor{sssec}{[check and remove the duplicate/multiple entries]}  

\item The bib files should not contain regular text such as notes. All items should belong to allowed database entries and fields \textcolor{sssec}{[check and remove the text entries or comment them out using \%]}  

\item Proper title case will not appear in the output (e.g., proper nouns like country names) even though things may seem right in \texttt{*.bib} file \textcolor{sssec}{[use additional curly braces \{\ldots\} to enclose the desired title case items or the letters]}. This happens especially with the title entries of the \texttt{*.bib}, but the journal entry will be fine.  

\item Accented characters, usually found in authors' names, might cause issues  \textcolor{sssec}{[correct use of codes to produce accented characters should be used and it is a good idea to enclose them with additional braces (e.g., \'e, \`o, \^e, \"o, \c c, \~n, \l, \L, \ae, \aa, through  \texttt{\textcolor{sssec}{\{\tb 'e\}, \{\tb `o\},  \{\tb \textasciicircum e\}, \{\tb"o\}, \{\tb c c\}, \{\tb\textasciitilde n\}, \{\tb l\}, \{\tb L\}, \{\tb ae\}, \textnormal{and} \{\tb aa\}}})]} 

\item Proper italics will not show in the output (\emph{Scientific names} of animals and plants; e.g., \emph{Homo sapiens, Zea mays}) \textcolor{sssec}{[use \cmd{emph\{\ldots\}} around the entries in the \texttt{*.bib} file]} 

\item Proper formatting of URLs with wrapping content and clickable links does not appear \textcolor{sssec}{[use \cmd{url\{\ldots\}} command in the \texttt{*.bib} files entries that have the URLs]} 

\item Outputs not formatted according to the thesis/journal requirement \textcolor{sssec}{[select the correct \blt style option or \texttt{bst} file that matches most of the requirements and modify the code to suit our requirement might sometimes require---most often only a few alterations are needed and have the file in the same folder as \texttt{*.bib}]} 

\item New bibliography styles that are not listed (Sec.~~\ref{bibgen}) might not run properly, mostly because of not having the corresponding \blt style or \texttt{*.bst} in the system \textcolor{sssec}{[download style files and have it in the system/local folder]}

\item Some numerical styles (e.g., IEEEtran family) will not work when the \texttt{natbib} was set to author-year format \textcolor{sssec}{[optional arguments of \texttt{natbib} package or arguments of \cmd{citestyle\{\}} command should be made compatible to the numerical styles (e.g., ``numbers'' or ``plain'' in their respective commands)---so a combination of chosen \texttt{*.bst} and  \texttt{citestyle option} is the key to create the references listing in the correct format]} 

\item Numerical bibliography styles reference listings with left-aligned numbers. \textcolor{sssec}{[See the note on the Sec.~\ref{bibgen}]}
 
\item While trying different styles, compilation stops even for compatible styles \textcolor{sssec}{[this is a common phenomenon and could be solved with an understanding of how \bt and \LaTeX{} resolves references by creating several auxiliary files (e.g., *.aux, *.toc, *.log, *.out) including *.bbl---as contents of these files were referred from unsuccessful compilation, removing all these files leaving the source (\texttt{*.tex}) and recompiling with the compatible style and options will restore and regenerate the output. Thus, starting with only a clean source will work and regenerate the necessary files, including the output; this strategy will work in other situations of compiling issues as well---however, it is not always necessary]}.
\end{itemize}

%--------------------------------------------------------------
\subsection{Proper development of bibliography bib file}
\vspace{-1ex}
\begin{itemize}[itemsep=0em, leftmargin=*]
\item Bib files should not be considered a dumping ground for bibliography codes that were usually not looked at for correctness, unlike the regular document source codes. This sometimes proves to be a serious mistake and stops compilation or produces strange outputs or major errors.

\item The bib file, therefore, should be developed step-by-step carefully, and each entry should be checked for correctness (only with ASCII characters and allowed equivalent \LaTeX{} commands for non-ASCII characters) and appropriateness (required and optional fields in bibliography entry). 

\item Bib files are automatically generated by other reference management software (e.g., Zotero) for a given set of references, while they are helpful and provide a good starting point, they are sometimes sources of compilation errors. This software may include ``abstract'' fields as well in the bib file, which are long and become sources of several unintended non-ASCII characters. These superfluous fields may be deleted so that the bib file is compact. Some of these software may include modern entries and fields that may be recognized by \blt but not included in \bt. Though it may be possible to create simpler and cleaner bib files, it is a good idea to check the generated bib file ``line-by-line'' for correctness.  

\item It was found that ``Google Scholar'' generated \bt entries worked well in developing the bib file. These entries usually generate logical naming of the keys and valid \LaTeX{} commands for non-ASCII characters (especially international accented characters). Even in these entries, checking the bib entries is always a good idea (e.g., use of capital letters and emphasis).

\item Common errors include:

\begin{itemize}
\item Extraneous non-ASCII characters (see Sec.~\ref{bibissue} first bullet);

\item \% sign - which is an innocent symbol, but with LaTeX anything after will be considered as a comment, so to produce \% we need to use \tb\%;

\item On a similar note, we have a set of reserved symbols (\&, \%, \$, \{, \}, \@, \_, \#, and so on) that carry a special meaning and should not be used (without ``\tb'' preceding; e.g., \tb\&, \tb\%, \tb\$, and so on) loosely and expect a logical outcome;

\item Every field (author, year, journal, etc.) should end with a comma (but the last one) - in a good bib file, just remove one comma and see the effect, which will teach us how a small thing will wreck the whole compilation;

\item Unconventional fields (though created by other software) should be replaced by the correct ones (Wikipedia \bt), for example, date instead of the year will not work;

\item Special symbols like degrees, copyright, and others commonly used in other systems can be coded in LaTeX as regular commands - and that command version of these symbols should be used in a bib or tex file; 

\item Duplicate entries. 

\item Unmatched braces. 

\item Several types of \bt entries (\url{https://en.wikipedia.org/wiki/BibTeX#Basic_structure}; e.g., \texttt{article, book, inproceedings, masterthesis, phdthesis, misc}, and so on) with their specific \emph{Required fields} and \emph{Optional fields}. Missing a required field (though generated by other resources) will be a mistake and will suspend compilation. The entry \emph{misc} does not contain any required field and can be used to fit odd entries.  

\item The foreign accented characters (e.g., \'e, \`o, \^e, \"o, \c c, \~n, \l, \L, \ae, \aa, and so on, that are created by \verb|\'e, \`o, \^e, \"o, \c c, \~n, \l, \L, \ae, \aa| \LaTeX{} commands) (check the LaTeX cheat sheet and learn how to code them - all start with \ followed by some ASCII symbols or characters and the input character that receives the accent. As a practice, try to have these accented characters in the tex file. If it runs there, then that can be inserted into the bib file. 
\end{itemize}

\item While building your bib file, add a couple of entries and check how they compile. Don't go for the whole 120 entries (just to throw a good number).

\item To visualize the reference listing of all the entries of the \texttt{*.bib}file issue the \verb|\nocite{*}| command. After the correctness of the entries is checked and fixed, the command may be removed, and the listing of only cited references will appear---nothing more, nothing less.

\item Further, remember the use of enclosing capital letters, especially in titles, for example, {NDSU}, {A}rgentina to get the capitals right from bib files. Similarly, the use of \cmd{emph\{\ldots\}} for technical names in bib files for italicized output (e.g., \emph{Zea~mays} for corn) - otherwise they come out in normal font.

\item It is a good idea while working with bib entries or in the main text, the liberal use of the \texttt{comment} environment, from the \texttt{comment} package, in the main text helps to check problematic bib entries and also produces efficient compilation.

%\item Automatic finding and replacement of non-ASCII (online tools or unix command line methods can be used).

\end{itemize}


%-------------------------------------------
\subsection{Chemical symbols}\index{chemical symbols}\index{chemical formula}
Chemical symbols and chemical equations can be coded easily in a natural manner using the \cmd{ch\{\ldots\}} command using the \texttt{chemformula} package---rather than using the math mode. The following chemicals: \ch{H2O}, \ch{H2SO4}, \ch{CrO4^2-}, \ch{[AgCl2]-}, \ch{(NH4)2S}, \ch{^{227}_{90}Th+}, and \ch{KCr(SO4)2 * 12 H2O} were coded through: \cmd{ch\{H2O\}}, \cmd{ch\{H2SO4\}}, \cmd{ch\{CrO4\textasciicircum2-\}}, \cmd{ch\{[AgCl2]-\}}, \cmd{ch\{(NH4)2S\}}, \cmd{ch\{\textasciicircum\{227\}\textunderscore\{90\}Th+\}}, and \cmd{ch\{KCr(SO4)2 * 12 H2O\}}, respectively.

\vspace{-2ex}
\begin{center}
\cmd{ch\{A + B ->[a] C\}} \hspace{1cm} gives \hspace{1cm} \ch{A + B ->[a] C}\\
\cmd{ch\{N2 + 3 H2 -> 2 NH3\}} \hspace{0.65cm} gives \hspace{1cm} \ch{N2 + 3 H2 -> 2 NH3}
\end{center}
Refer to \texttt{chemformula} documentation for more options and details. It should be noted that there are other packages available for coding the chemicals and chemical equations, which are not included in the class, but users can use them through \cmd{usepackage}\{\ldots\} command. 


%%--------------------------------------------------------------
\subsection{Individual or combine reference listing}\index{reference listing}
The two types of reference listings are in the individual chapters, added that the end or a combined reference listing for all chapters as a standalone unnumbered chapter. 

%-------------------------------------------
\subsection{Annotation commands}
While developing the dissertation, the text undergoes several revisions, and suggestions will be provided by the advisor and colleagues. To make suggestions as well as to present the carried out revisions, colored annotations will be helpful to draw users' attention quickly. Therefore, special annotation commands for \ix{highlighting}, \ix{new text}, \ix{deleted text}, \ix{replaced text}, and notes were defined in the class. These annotation features can be used by the student and the advisor while reviewing the dissertation draft. The \texttt{ulem} and \texttt{\ix{todonotes}} packages were used to develop these commands, and their documentation may be referred to for customization. All the \ix{annotations} can be searched and deleted before submission, and these processes can be even automated by search expressions (e.g., regular expressions). The annotation commands with usage are shown subsequently:

\vspace{2ex}
\tb\texttt{hl\{Highlight\}} gives: \hl{Highlight}. This will be regular text.

\tb\texttt{nt\{Test new text.\}} gives: \nt{Test new text.} This will be regular text.

\tb\texttt{dt\{Deleted text.\}} gives: \dt{Deleted text.} This will be regular text.

\tb\texttt{rt\{The text to be deleted\}\{Which will be replaced by this!\}} gives:

\rt{The text to be deleted}{Which will be replaced by this!} This will be a regular text again.

\vspace{2ex}
While using the above annotation commands, except for \cmd{nt\{\ldots\}}, enclosing a cited reference commands (\cmd{citep\{\ldots\}} or \cmd{citet\{\ldots\}}) use \cmd{mbox
\{\ldots\}} around the cited references. For example, \\\cmd{dt\{\ldots text\ldots \cmd{mbox\{\cmd{citep\{daly2010natural\}}\}} \ldots text\ldots\}} 
gives: \dt{\ldots text\ldots (Daly, 2010) \ldots text \ldots} 


\vspace{2ex}
\tb\texttt{notes\{To Do notes - for interactive communication!\}} (also the shortcut \cmd{td\{\ldots\}}) gives: \notes{To Do notes - for interactive communication!} 



%--------------------------------------------------------------
\subsection{Clever reference---\ix{cross-referencing} items and labels}
Referring to items automatically using the defined labels is a common activity in \LaTeX\ and is called cross-referencing. Although there are basic commands available to refer (e.g., fig.~\cmd{ref\{label\}}), the use of \texttt{\ix{cleveref}} package is an efficient way to achieve this task. This package enhances \LaTeX's cross-referencing to automatically detect the ``type'' of the item cross-referenced (e.g., equation, section, tables, figures, schemes, etc.) based on the context of the cross-reference. This means a single command of \cmd{cref\{label\}} or \cmd{Cref\{label\}} with the label will produce the correct output (e.g., fig.~1.1, eq.~3, Figure~1.1, Equation~3, etc.). Refer to this package for more details and customization. However, \texttt{cleveref} commands will not work with the appendix tables and appendix figures only, where the basic commands (\cmd{ref\{\ldots\}}) should be used. \index{cref} \index{Cref}


%-------------------------------------------
\vspace{-1.77ex}
\subsection{Chapters as \ix{individual files}}
When the length of chapters gets long, it will be better managed into individual \texttt{*.tex} files. Then the thesis file will become a collection of such individual files and will be highly compact. The individual chapters are coded using either of these commands:
\ccmd{
\cmd{\ix{input}\{filename.tex\}}\\
\cmd{\ix{include}\{filename\}}\\
}

The \cmd{input\{filename.tex\}} \index{input file} imports the codes from the \texttt{filename.tex} into the main file at the location where this command was issued. This is equivalent to typing all the code commands from the individual file into the main file. The \cmd{include\{filename\}} issues a \cmd{clearpage} before and after inserting the contents and has better speed than the \cmd{input\{filename.tex\}} command. With such commands in place, it is possible to compile only the chapter the user wants to work on by commenting out others, and this approach saves unnecessary compilation or reduces \ix{compilation times}. 

This procedure helps in focusing on the ``work at hand'' and enhances writing quality and productivity. It should be noted that the entire code of the document is ``there'' but only hidden from compiling, and the comment commands can be removed or ``commented'' to produce the entire document at any time for the final use. One drawback of this approach is that all the numbering (e.g., chapter, section, tables, figures) will be reset and start from the beginning, and may not represent the final whole document's actual numbering. Just to address this situation, the concept of \texttt{includeonly} was developed (see Sec.~\ref{inclonly}).  

Furthermore, these individual chapter files can be stored under a folder (similar to \texttt{graphicspath}. For example, use the command below that declares the path of the folder (relative to the default directory) where the source codes of all chapters (\texttt{*.tex}) can be kept inside the folder of the default directory (e.g. \texttt{mychapters}). 

\ccmd{\cmd{chapterspath\{./mychapters/\}}}

With this method, the default directory (or project folder) will contain only one project tex file (e.g., \texttt{main.tex}), which calls the other chapters and figures from the respective folders, and other resources necessary for building the output (\texttt{cls, bib, bst, \ldots}). This way, all the figures and chapters will be ``tugged away'' and promote better file organization (refer to \texttt{sample thesis includeOnly} project included with the class). In such a system of storing the files separately, only the main project file can be compiled, which uses the codes of chapters in other folders, to generate the output, and the individual chapters cannot be compiled directly as the code does not contain the \texttt{documentclass}, prefatory material, and \texttt{document} environment---which is obvious.  

%-------------------------------------------
%\vspace{-1.77ex}
\subsection{Chapters using ``\ix{includeonly}'' method---efficient working and compiling}\label{inclonly}
\LaTeX{} allows for developing the document piece-by-piece (e.g., individual chapters, sections, and even paragraphs), currently being worked on. One way of doing this is to use the comments package's \texttt{comment} environment. The portion of the source code, however long, enclosed between \cmd{begin\{comment\}} and \cmd{end\{comment\}} will be excluded from compilation and output. This will often be convenient than commenting on individual lines and paragraphs using \% character. 

On the same token, applying two comment blocks, above and below the current chunk of code (e.g., chapter, section, and so on), obviously leaving the \cmd{begin\{document\}} and \cmd{end\{document\}} lines, the only portion of the code will be compiled and displayed. The \cmd{end\{comment\}} of the top block and the \cmd{begin\{comment\}} of the end block can be moved as required to compile and output different portions of the code for display and inspection. An alternative method is to replace the bottom comment block with the direct \cmd{tend} or other variations (temporary ending command; see Sec.~\ref{temend}). This method will end the document and only display the compiled output after the first comment block and before the temporary end command. 

As mentioned earlier, commenting out chapters (direct codes or included or input chapters) will reset the numbering system, which affects the natural order. This issue was solved by the method of ``\texttt{includeonly},'' where the numbering system is preserved, irrespective of commenting out the included chapters, and the approach of using this along with a main project file is as follows. The \cmd{includeonly} command coded in the preamble has all the chapters but the last (in this case \texttt{appendixB}). The last chapter should be added in the form of \cmd{input} command to suppress the blank last page created by \cmd{include} command, as shown.  

\setlength{\columnsep}{1cm}
\setlength{\columnseprule}{0.4pt}
\begin{multicols}{3}
\noindent\textcolor{sec!60!sssec}{\underline{First run}}
{\scriptsize
\begin{verbatim}
... Preamble ...
\includeonly{ 
chapter1, 
chapter2,
chapter3,
chapter4,
appendixA
}
... 
\begin{document}
... 
\myheading{Test Chapter for NDSU Thesis Class Sandbox}

\checkBeginRefsection%%% Don't delete this

This ``\texttt{ndsu-sandbox.tex}'' file can be used as a sandbox to try out things in the actual NDSU thesis environment. \textcolor{gray}{Things tested here (including the bibliography) can be readily inserted into the original thesis/dissertation document. Therefore, this lightweight source will be convenient to test things out.} So, go for it --- and remember anything is possible by \LaTeX{} (almost!?).

%----------------------------------------------------
\section{Section}
\subsection{Sub-Section}
\subsubsection{Sub-Sub-Section}

\textcolor{gray}{Dummy text from kantlipsum[9]. Reference listing on the next page. Check it for the intended formatting.} I refer to \citep{lamport94,kopka2004guide,baczkowski1990ndsu}. 

\begin{equation}
y = mx + c
\label{eq:line}
\end{equation}

The straight-line equation presented above (\cref{eq:line}) is the simplest.

\kant[9]

\textcolor{red}{Text in red.}

%\tendb
%\tendc

\begin{table}[ht]
\centering
\caption{Professional looking fixed-width table using 
\texttt{booktabs} package.}
\begin{tabular}{ l c r }
\toprule
Number & Our rating & Month \\
(left) & (center)   & (right)\\
\midrule
1 & Colder & January \\
2 & Okay   & February \\
3 & Good   & March\\
\bottomrule
\end{tabular}
\label{tab22}
\end{table}

\kant[9]

\begin{table}[h!]
\centering
\caption{Professional looking automatic full-width table using \texttt{tblr} environment and \texttt{booktabs} package.}
\begin{tblr}{X | X[c] | X[r]}
\toprule
Number & Our rating & Month \\
(left) & (center)   & (right)\\
\midrule
1 & Colder & January \\
2 & Okay   & February \\
3 & Good   & March\\
\bottomrule
\end{tblr}
\label{tab25}
\end{table}

\kant[9]

\myfig{H}{0.4}{frog.jpg}{Figure short caption is centered. 
Use of myfig command.}{fig2}


\kant[9]
\myfig{H}{0.8}{fig-LOA}{Figure short caption is centered. 
Use of myfig command; Now long caption that will be left-justified.}{fig3}

\kant[2-4]

%----------------------------------------------------
\section{No bibliography included}
If we don't issue the \hl{\textbackslash \texttt{makerefs}}, as done here below, the reference listing will not appear. This behavior, if needed, can be produced without the reference creation command --- the logic represents the basic behavior. 

\checkEndRefsection%%% Don't delete this
%----------------------------------------------------
%----------------------------------------------------
%----------------------------------------------------
 
\myheading{Test Second Chapter for NDSU Thesis Class Sandbox}

\checkBeginRefsection%%% Don't delete this

\kant[20-21]

%----------------------------------------------------
\section{Section}
\subsection{Sub-Section}
\subsubsection{Sub-Sub-Section}

\textcolor{gray}{Dummy text from kantlipsum. Reference listing on the next page. Check it for the intended formatting.} I refer to \citep{butin2009education, rudestam2014surviving, Goossens2008g,cassuto2010advising,pires2021teens}. 

\kant[9]

\myfig{H}{0.4}{frog}{Short caption is centered. Use of myfig command.}{fig4}

\begin{align}
y_1 & = mx + c_1 
\label{eq:line1} \\
y_2 & = mx + c_2 
\label{eq:line2} \\
y_3 & = mx + c_3 
\label{eq:line3}
\end{align}

A family of straight-line equations is presented above (\cref{eq:line1,eq:line2,eq:line3}).


\kant[10]
\myfig{H}{0.95}{fig-LOS}{Figure from the figures folder. Short caption is centered. Use of myfig command; Now long caption that will be left-justified.}{fig5}

\textcolor{blue}{Text in blue.}

%----------------------------------------------------
\kant[2-5]

\newpage

\checkEndRefsection%%% Don't delete this
%----------------------------------------------------
%----------------------------------------------------
%----------------------------------------------------

\include{chapter3}
\myheading{Test Fourth Chapter for NDSU Thesis Class Sandbox}

\checkBeginRefsection%%% Don't delete this
\kant[20-21]

%----------------------------------------------------
\section{Section}
\subsection{Sub-Section}
\subsubsection{Sub-Sub-Section}

\textcolor{gray}{Dummy text from kantlipsum. Reference listing on the next page. Check it for the intended formatting.} I refer to \citep{butin2009education, rudestam2014surviving, Goossens2008g,cassuto2010advising,pires2021teens}. 

\kant[9]

\myfig{H}{0.4}{frog}{Short caption is centered. Use of myfig command.}{fig4}

\kant[10]
\myfig{H}{0.95}{agFarm-Free}{Figure from the figures folder. Short caption is centered. Use of myfig command; Now long caption that will be left-justified.}{fig5}

\textcolor{green}{Text in GREEN.}

%----------------------------------------------------
\kant[2-6]

\checkEndRefsection%%% Don't delete this
%----------------------------------------------------
%----------------------------------------------------

%******************* Named appendix A *******************
\namedappendices{A}{Named first appendix}

\checkBeginRefsection%%% Don't delete this 

Appendix material can be included here. First include a figure (fig.~\ref{fig:ap1}).

%-------------------------------------------------
\section{Appendix A - Section With Figure}
\myfigap{H}{0.5}{example-image-golden}{A golden ratio rectangle image.}{fig:ap1}	\kant[8]

%-------------------------------------------------
\section{Appendix A - Section With Table}
And, then including a table (table.~\ref{tab:ap1}).

\begin{appendixtable}[h!]
\centering
\caption{Use of \texttt{tblr} environment for full-width table - applicable to both main text 
and appendix.  Note the use of \texttt{booktabs} commands and `X' parameters to reproduce 
Table~\ref{tab:ap1}.}
\begin{tblr}{*4X}
\toprule
Number 		& Name of month 	& Days 	& Season\\
\midrule
\#7 			& July       			& 30 		& Spring\\ \cmidrule[lr]{2-4}
Multicolumn 	&\SetCell[c=3]{c} The three columns combined \\ \cmidrule[lr]{2-4}
\#8 			& August 		   	& 31 		& Summer\\
\#9 			& September 		& 30 		& Summer\\
\bottomrule
\end{tblr}
\begin{tablenotes}[flushleft]
\footnotesize
\item \hspace{-1ex} \emph{Note}: The \texttt{tablenotes} environment produces table footnotes.  
Refer to \texttt{tabularray} documentation for further details.  
\end{tablenotes}
\label{tab:ap1}
\end{appendixtable}

%-------------------------------------------------
\subsection{Appendix A Subsection}

\textcolor{gray}{Dummy text from kantlipsum[9]. Reference listing on the next page. Check it for the intended formatting.} I refer to \citep{lamport94,kopka2004guide,baczkowski1990ndsu}. 

\myalign{
y_1 & = m_1x + c_1 
\label{eq:lineA1} \\
y_2 & = m_2x + c_2 
\label{eq:lineA2} \\
y_3 & = m_3x + c_3 
\label{eq:lineA3}
}

A family of straight-line equations is presented above (\cref{eq:lineA1,eq:lineA2,eq:lineA3}).

\kant[10]

\checkEndRefsection%%% Don't delete this

%******************* END *******************
%******************* Named Appendix B *******************
\namedappendices{B}{Named second appendix}

\checkBeginRefsection%%% Don't delete this

Appendix material can be included here. First include a figure (fig.~\ref{fig:ap2}).

%-------------------------------------------------
\section{Appendix B - Section With Figure}

If no article is cited, no reference listing will be made. The following sentence containing citations sentence was commented, hence no reference listing was generated in this appendix chapter.  
%\textcolor{gray}{Dummy text from kantlipsum. Reference listing on the next page. Check it for the intended formatting.} I refer to \citep{butin2009education,rudestam2014surviving}. 

\kant[9]

\myfracalign{
b_1 & = m_1x + \frac{c_1}{d} 
\label{eq:lineB1} \\
b_2 & = m_2x + \frac{c_2}{e}
\label{eq:lineB2}
}

A family of straight-line equations is presented above (\cref{eq:lineB1,eq:lineB2}).

\kant[13]

\myfigap[0.5ex]{H}{0.6}{example-grid-100x100pt}{A $10 \times 10$ grid of different
concentric colors.}{fig:ap2}

\section{Appendix B - Section With Table}
Now coding another appendix table (table.~\ref{tab:ap2}) that spans the entire width using
the manual method (using `tabcolsep' command; and `resize' command to fit large tables).

\begin{appendixtable}[h]
\centering
\caption{Squares and cubes named appendix table using \texttt{siunitx} and \texttt{tabularray} 
packages.}
\begin{tblr}{X X[c] X[r] X[1.5,r]}
\toprule
Number 	& Square        		& Cubes          		& Fourth power\\
\midrule
11 	   	& 121   			& \num{1331} 		& \num{146412}\\
22 	   	& 484  			& \num{10648}		& \num{234256}\\
333 	  	& \num{110889}  	& \num{36926037}	& \num{12296370321}\\
\bottomrule
\end{tblr}
\label{tab:ap2}
\end{appendixtable}

%-------------------------------------------------
\subsection{Appendix B Subsection}
\kant[11-12]

\checkEndRefsection%%% Don't delete this

%******************* END *******************
... 
\closeappendices
\end{document}
\end{verbatim}
}

\columnbreak
\noindent\textcolor{sec!60!sssec}{\underline{Second and subsequent runs}}
{\scriptsize
\begin{verbatim}
... Preamble ...
\includeonly{ 
chapter1, 
%chapter2,
%chapter3,
chapter4,
appendixA
}
... 
\begin{document}
... 
\myheading{Test Chapter for NDSU Thesis Class Sandbox}

\checkBeginRefsection%%% Don't delete this

This ``\texttt{ndsu-sandbox.tex}'' file can be used as a sandbox to try out things in the actual NDSU thesis environment. \textcolor{gray}{Things tested here (including the bibliography) can be readily inserted into the original thesis/dissertation document. Therefore, this lightweight source will be convenient to test things out.} So, go for it --- and remember anything is possible by \LaTeX{} (almost!?).

%----------------------------------------------------
\section{Section}
\subsection{Sub-Section}
\subsubsection{Sub-Sub-Section}

\textcolor{gray}{Dummy text from kantlipsum[9]. Reference listing on the next page. Check it for the intended formatting.} I refer to \citep{lamport94,kopka2004guide,baczkowski1990ndsu}. 

\begin{equation}
y = mx + c
\label{eq:line}
\end{equation}

The straight-line equation presented above (\cref{eq:line}) is the simplest.

\kant[9]

\textcolor{red}{Text in red.}

%\tendb
%\tendc

\begin{table}[ht]
\centering
\caption{Professional looking fixed-width table using 
\texttt{booktabs} package.}
\begin{tabular}{ l c r }
\toprule
Number & Our rating & Month \\
(left) & (center)   & (right)\\
\midrule
1 & Colder & January \\
2 & Okay   & February \\
3 & Good   & March\\
\bottomrule
\end{tabular}
\label{tab22}
\end{table}

\kant[9]

\begin{table}[h!]
\centering
\caption{Professional looking automatic full-width table using \texttt{tblr} environment and \texttt{booktabs} package.}
\begin{tblr}{X | X[c] | X[r]}
\toprule
Number & Our rating & Month \\
(left) & (center)   & (right)\\
\midrule
1 & Colder & January \\
2 & Okay   & February \\
3 & Good   & March\\
\bottomrule
\end{tblr}
\label{tab25}
\end{table}

\kant[9]

\myfig{H}{0.4}{frog.jpg}{Figure short caption is centered. 
Use of myfig command.}{fig2}


\kant[9]
\myfig{H}{0.8}{fig-LOA}{Figure short caption is centered. 
Use of myfig command; Now long caption that will be left-justified.}{fig3}

\kant[2-4]

%----------------------------------------------------
\section{No bibliography included}
If we don't issue the \hl{\textbackslash \texttt{makerefs}}, as done here below, the reference listing will not appear. This behavior, if needed, can be produced without the reference creation command --- the logic represents the basic behavior. 

\checkEndRefsection%%% Don't delete this
%----------------------------------------------------
%----------------------------------------------------
%----------------------------------------------------
 
\myheading{Test Second Chapter for NDSU Thesis Class Sandbox}

\checkBeginRefsection%%% Don't delete this

\kant[20-21]

%----------------------------------------------------
\section{Section}
\subsection{Sub-Section}
\subsubsection{Sub-Sub-Section}

\textcolor{gray}{Dummy text from kantlipsum. Reference listing on the next page. Check it for the intended formatting.} I refer to \citep{butin2009education, rudestam2014surviving, Goossens2008g,cassuto2010advising,pires2021teens}. 

\kant[9]

\myfig{H}{0.4}{frog}{Short caption is centered. Use of myfig command.}{fig4}

\begin{align}
y_1 & = mx + c_1 
\label{eq:line1} \\
y_2 & = mx + c_2 
\label{eq:line2} \\
y_3 & = mx + c_3 
\label{eq:line3}
\end{align}

A family of straight-line equations is presented above (\cref{eq:line1,eq:line2,eq:line3}).


\kant[10]
\myfig{H}{0.95}{fig-LOS}{Figure from the figures folder. Short caption is centered. Use of myfig command; Now long caption that will be left-justified.}{fig5}

\textcolor{blue}{Text in blue.}

%----------------------------------------------------
\kant[2-5]

\newpage

\checkEndRefsection%%% Don't delete this
%----------------------------------------------------
%----------------------------------------------------
%----------------------------------------------------

\include{chapter3}
\myheading{Test Fourth Chapter for NDSU Thesis Class Sandbox}

\checkBeginRefsection%%% Don't delete this
\kant[20-21]

%----------------------------------------------------
\section{Section}
\subsection{Sub-Section}
\subsubsection{Sub-Sub-Section}

\textcolor{gray}{Dummy text from kantlipsum. Reference listing on the next page. Check it for the intended formatting.} I refer to \citep{butin2009education, rudestam2014surviving, Goossens2008g,cassuto2010advising,pires2021teens}. 

\kant[9]

\myfig{H}{0.4}{frog}{Short caption is centered. Use of myfig command.}{fig4}

\kant[10]
\myfig{H}{0.95}{agFarm-Free}{Figure from the figures folder. Short caption is centered. Use of myfig command; Now long caption that will be left-justified.}{fig5}

\textcolor{green}{Text in GREEN.}

%----------------------------------------------------
\kant[2-6]

\checkEndRefsection%%% Don't delete this
%----------------------------------------------------
%----------------------------------------------------

%******************* Named appendix A *******************
\namedappendices{A}{Named first appendix}

\checkBeginRefsection%%% Don't delete this 

Appendix material can be included here. First include a figure (fig.~\ref{fig:ap1}).

%-------------------------------------------------
\section{Appendix A - Section With Figure}
\myfigap{H}{0.5}{example-image-golden}{A golden ratio rectangle image.}{fig:ap1}	\kant[8]

%-------------------------------------------------
\section{Appendix A - Section With Table}
And, then including a table (table.~\ref{tab:ap1}).

\begin{appendixtable}[h!]
\centering
\caption{Use of \texttt{tblr} environment for full-width table - applicable to both main text 
and appendix.  Note the use of \texttt{booktabs} commands and `X' parameters to reproduce 
Table~\ref{tab:ap1}.}
\begin{tblr}{*4X}
\toprule
Number 		& Name of month 	& Days 	& Season\\
\midrule
\#7 			& July       			& 30 		& Spring\\ \cmidrule[lr]{2-4}
Multicolumn 	&\SetCell[c=3]{c} The three columns combined \\ \cmidrule[lr]{2-4}
\#8 			& August 		   	& 31 		& Summer\\
\#9 			& September 		& 30 		& Summer\\
\bottomrule
\end{tblr}
\begin{tablenotes}[flushleft]
\footnotesize
\item \hspace{-1ex} \emph{Note}: The \texttt{tablenotes} environment produces table footnotes.  
Refer to \texttt{tabularray} documentation for further details.  
\end{tablenotes}
\label{tab:ap1}
\end{appendixtable}

%-------------------------------------------------
\subsection{Appendix A Subsection}

\textcolor{gray}{Dummy text from kantlipsum[9]. Reference listing on the next page. Check it for the intended formatting.} I refer to \citep{lamport94,kopka2004guide,baczkowski1990ndsu}. 

\myalign{
y_1 & = m_1x + c_1 
\label{eq:lineA1} \\
y_2 & = m_2x + c_2 
\label{eq:lineA2} \\
y_3 & = m_3x + c_3 
\label{eq:lineA3}
}

A family of straight-line equations is presented above (\cref{eq:lineA1,eq:lineA2,eq:lineA3}).

\kant[10]

\checkEndRefsection%%% Don't delete this

%******************* END *******************
%******************* Named Appendix B *******************
\namedappendices{B}{Named second appendix}

\checkBeginRefsection%%% Don't delete this

Appendix material can be included here. First include a figure (fig.~\ref{fig:ap2}).

%-------------------------------------------------
\section{Appendix B - Section With Figure}

If no article is cited, no reference listing will be made. The following sentence containing citations sentence was commented, hence no reference listing was generated in this appendix chapter.  
%\textcolor{gray}{Dummy text from kantlipsum. Reference listing on the next page. Check it for the intended formatting.} I refer to \citep{butin2009education,rudestam2014surviving}. 

\kant[9]

\myfracalign{
b_1 & = m_1x + \frac{c_1}{d} 
\label{eq:lineB1} \\
b_2 & = m_2x + \frac{c_2}{e}
\label{eq:lineB2}
}

A family of straight-line equations is presented above (\cref{eq:lineB1,eq:lineB2}).

\kant[13]

\myfigap[0.5ex]{H}{0.6}{example-grid-100x100pt}{A $10 \times 10$ grid of different
concentric colors.}{fig:ap2}

\section{Appendix B - Section With Table}
Now coding another appendix table (table.~\ref{tab:ap2}) that spans the entire width using
the manual method (using `tabcolsep' command; and `resize' command to fit large tables).

\begin{appendixtable}[h]
\centering
\caption{Squares and cubes named appendix table using \texttt{siunitx} and \texttt{tabularray} 
packages.}
\begin{tblr}{X X[c] X[r] X[1.5,r]}
\toprule
Number 	& Square        		& Cubes          		& Fourth power\\
\midrule
11 	   	& 121   			& \num{1331} 		& \num{146412}\\
22 	   	& 484  			& \num{10648}		& \num{234256}\\
333 	  	& \num{110889}  	& \num{36926037}	& \num{12296370321}\\
\bottomrule
\end{tblr}
\label{tab:ap2}
\end{appendixtable}

%-------------------------------------------------
\subsection{Appendix B Subsection}
\kant[11-12]

\checkEndRefsection%%% Don't delete this

%******************* END *******************
... 
\closeappendices
\end{document}
\end{verbatim}
}

\columnbreak
\noindent\textcolor{sec!60!sssec}{\underline{Notes}}

\vspace{1ex}
{\scriptsize
\noindent The number of chapters/files in the \texttt{includeonly\{\:\}} should match the files used in the \cmd{include\{\ldots\}} command. All chapter lines of code being active in the first run will create the \texttt{*.tex} files. In the second run shown only chapters 1, 4, and Appendix A will be output. Any or all chapters in the \texttt{includeonly\{\:\}} can be made inactive or active in any subsequent runs, and corresponding output will be generated. The total thesis TOC and other prefatory contents will always be generated irrespective of the selection of chapters.
} 
\end{multicols}

For the initial run, all lines of code with \texttt{includeonly} and \texttt{include} \index{include files} of the chapters should be active, which will produce the necessary working files (*.aux) for all chapters. In the later runs only the required chapters in the \texttt{includeonly} portion of the code can be kept active (other chapters commented, but ``\texttt{include}'' chapters always kept active---refer to the additional example \texttt{sample thesis includeOnly} project in the class bundle) and will produce the output of those active chapters only---with all contents correctly numbered. Another advanced method is to use \texttt{subfiles}, where individual chapters can be directly compiled. \index{input file}

%--------------------------------------------------------------
\subsection{Temporary ending}{\label{temend}}
While reviewing/revising a large document, it will be efficient to compile only the chapter/section that is currently being worked on. The temporary ending command coded in the class, when issued, will generate a shorter document while the rest of the document will be ignored (Table~\ref{tends}). As this command only makes a temporary ending, all the source codes from the beginning until the command will appear in the output. To focus and work on a specific chapter/section/paragraph and so on, a combination of the ``includeonly'' (Sec.~\ref{inclonly}) and/or \texttt{comment} environment should be used with the temporary ending. Therefore, using this method,  it is possible to work on a single paragraph of source text.

\vspace{-1ex}
\begin{table}[h!]
\footnotesize
\centering
\caption{Temporary ending commands for different document scenarios and their usage}
\vspace{-1.5ex}
\begin{tabular}{l l @{\hspace{4mm}} p{3.3in}}
\toprule
Document scenario & Command & Outcomes \\
\midrule
Single/multiple files	& \texttt{\cmd{\ix{tend}}} & Ends document - should not have \texttt{[chapterrefs]} option in \\
& &  documentclass and no reference listing produced\\
\cmidrule[0.2pt](l{1.5ex}r{1.5ex}){1-3}
Single/multiple files, \bt \& who. refs. & \texttt{\cmd{\ix{tendbt}}} & Ends document - makes reference listing using \bt style\\[1ex]
Single/multiple files, \blt \& who. refs.  & \texttt{\cmd{\ix{tendblt}}} & Ends document - makes reference listing using \blt resource,\\
& & produce chapter individual reference listing with \texttt{[chapterrefs]} option otherwise a single whole reference so far\\
\bottomrule
\end{tabular}
\begin{tablenotes}[flushleft]
\footnotesize
\item \hspace{-1ex} \emph{Note}: who. refs. - whole document single reference chapter.
\end{tablenotes}
\label{tends}
\end{table}

Since we deal with source codes having different features (\bt \emph{vs} \blt, individual chapter \emph{vs} whole document reference, single \emph{vs} multiple source files), based on the users' choice, several temporary ending commands were developed (Table~\ref{tends}) with usage. Based on the type, these commands will end the document with or without a reference listing. These commands have instructions to appropriately close the respective environments (e.g., \texttt{refsection, document}), and users should ensure that the respective environment is active before these commands.  It should be noted that the temporary ending commands for \bt and \blt should not be applied interchangeably.  


%-------------------------------------------
\subsection{Defining and using \ix{specific commands}, environments, and packages}
As it is not possible to write a class to satisfy the specific requirements of several departments of NDSU, most of the major features as outlined in this document were coded into the NDSU class, and the users can add necessary features in their source code specific to their requirements. This approach gives flexibility, making the class compact and useful. Mathematics, physics, chemistry, literature, etc., departments may use specific symbols and environments that other departments (e.g., agriculture) do not use for their thesis. This means that based on the requirement, if the packages are not already loaded in the class, the necessary packages can be loaded in the source files, and they work with the class. Another aspect is the bibliography style and management, which varies with different departments. To deal with this, the necessary style files (\texttt{*.bst}) should be kept in the same folder as \texttt{*.tex} or the appropriate path specified in the source code.

Specific packages (e.g., bibliography management), \ix{new commands} (shortcuts), and new environments (e.g., theorem, proof, etc.) can be included in the preamble. These specific items are best left to individual users, as others may not need this---hence they are not included in the class deliberately. \index{specific packages}

%-------------------------------------------
\subsection{Thesis to \ix{journal article} conversion for submission}\index{thesis to paper}
A few adjustments are necessary to convert the thesis chapters into journal articles. The commands that are specific to the class (e.g., \cmd{myfig\{\}}, \cmd{mysch\{\}}, etc.) will not be known to the journal \LaTeX\ template. When the corresponding \texttt{newcommand} codes (made of basic commands), as shown in Table~\ref{tab1}, against the class shortcut below were inserted in the template preamble for the source to compile. As the shortcuts are convenient and reduce the amount of code typed (Don't repeat yourself---\ix{DRY} method), it will be better to insert the preamble and retain the thesis code than to replace it with basic commands at all occurrences. 

\begin{table}[ht!]
\centering
\captionsetup{singlelinecheck=false}
\caption{\noindent Converting thesis chapters to journal articles---preamble information to be added to journal template source code.}
\vspace{-1ex}
\begin{tabular}{ p{2in}  p{4.15in} }
\toprule
{\small \textbf{Shortcut example commands in thesis} (Refer Sec. \ref{figureshortcut})} & {\small\textbf{Command to be inserted in the preamble for journal articles}} \\
\midrule
{\small \vspace{-5.6em} 
\verb+\myfig[0.4cm]+
\verb+{h!}{1.0}{fig1.png}+
\verb+{Regular figure caption}+
\verb+{fig:1}+ 
}&{\small
\begin{minipage}{4in}
\begin{verbatim}
\newcommand{\myfig}[6][0ex]{
\begin{figure}[#2]
\centering 
\captionsetup{aboveskip=#1 plus 2pt minus 2pt, belowskip=0pt} 
\includegraphics[width=#3\textwidth]{#4}
\caption{#5}
\label{#6}
\end{figure}
}
\end{verbatim}
\end{minipage}}\\
\midrule

{\small \vspace{-6.8em}
\verb+\myfigls[0.5cm]+
\verb+{ht}{0.8}{fig2.pdf}+
\verb+{Landscape figure caption}+
\verb+{figls:2}+ 
}&{\small
\begin{minipage}{4in}
\begin{verbatim}
\newcommand{\myfigls}[6][0ex]{
\begin{landscape}
\begin{figure}[#2]
\centering 
\captionsetup{aboveskip=#1 plus 2pt min?us 2pt, belowskip=0pt}
\includegraphics[width=#3\textwidth]{#4}
\caption{#5}
\label{#6}
\end{figure}
\end{landscape}
}
\end{verbatim}
\end{minipage}}\\
\bottomrule
\multicolumn{2}{l}{\footnotesize Note: Also to be followed with scheme shortcuts when employed}
\end{tabular}
\label{tab1}
\end{table}

The compatibility issue will not be there with the tables, for no shortcuts were defined in the class, and all table codes used the basic commands from the respective packages. Also, the heading commands (\cmd{myheading\{\}} and \cmd{mypaperheading\{\}}) are specific to the thesis class (see Sec.~\ref{heading}) and will not generally work with the journal template. However, their different arguments can be extracted and used in the commands specific to the journal template. Similarly, the code from the appendices, such as \texttt{appendixtable} and \texttt{appendixfigure} environments, should be replaced by the basic \texttt{table} and \texttt{figure} environments, respectively, in the article template. In general, a paper will not have schematics, and those are usually coded as figures. Needless to mention, the NDSU class-specific front matter commands that may not appear in individual chapters, such as \texttt{author, date, progdeptchoice, department, cchair, cmember(a-d), approvaldate, approver, listofabbreviations, listofsymbols}, and so on, will not work in the journal template; but the content of the arguments can be copied and utilized in the appropriate journal template commands. 

It may also be necessary to include the missing packages in the template, which were used in the thesis class. The missing packages can be identified from the ``\texttt{Undefined control sequence}'' error messages that stop the compilation (e.g., using \cmd{cref\{\}} without loading the \texttt{cleveref} package).  Of course, the template-specific commands should be used while developing journal articles. With these adjustments, the thesis code easily transitions to the journal articles. It is hoped that with the working knowledge of \LaTeX\ previously obtained and/or gathered thus far, the students will be able to handle what is required to convert the thesis chapters into journal articles' source code utilizing the specific journal templates. 

%--------------------------------------------------------------
%--------------------------------------------------------------
\section{Additional Information \textrm{II}---Some Tips For Customization}

%-------------------------------------------
\vspace{-1ex}
\subsection{General suggestions}
NDSU Graduate School Formatting Guidelines (
\url{https://www.ndsu.edu/sites/default/files/2021-09/Format-Guidelines-2021.pdf}) should be adhered to, and the guidelines to be referred to for various aspects of developing the work. Following are some of the \ix{general suggestions} while developing the thesis/dissertation:
\begin{itemize}[itemsep=0em, leftmargin=*]
\item
All content-related decisions should be made by the student, advisor, and committee, and should follow any rules or conventions established within your program, department, or field.
\item
Students are highly encouraged to follow the prevalent style manual of the discipline (e.g., MLA, APA, Chicago, IEEE, etc.) while formatting the thesis (especially formatting of tables, figures, and other non-textual elements). In the instances where the Graduate School guidelines contradict the style manual for your discipline, the former takes precedence. However, if a generic style is applied consistently throughout all items, it will also be approved. During format review, a consistent application of one style is accepted. In short, consistency is KEY.
\item
Paper or regular thesis/dissertation chapters (two possible styles) should follow NDSU format guidelines consistently across all chapters and use the prevalent \ix{style manual} of the discipline.
\item
References, tables, and figures should follow the most appropriate style manual of the discipline. Some have the caption centered and set in bold font. 
\item
A footnote should be included if the chapter is \ix{co-authored} (an example of this is in NDSU guidelines), and including publication information in the footnote or in the Acknowledgments section is recommended.
\item
The general recommendation about spacing between items and their surrounding non-textual content (equations, figures, tables, quotes, pseudocode, etc.) is to set a consistent spacing between items and their surrounding content, as seen in most academic publications.
\item
Table (and figure) captions should be in the same font size as general text; however, text inside tables and footnotes may be in smaller font sizes as needed to fit the item within the page margins.
\item
NDSU guidelines have a number of very specific rules (e.g., Table of Contents and Prefatory List formatting, abstract word count, headings, body text paragraphs, etc.); however, they give a lot of flexibility for what is not covered, and give the student (and committee) control of the written content.
\item
It should be noted that \LaTeX\ is a vast program with numerous facilities and resources that students can use while developing their thesis/dissertation and improving the document quality. All regular \LaTeX\ commands and features work well with the NDSU thesis class. 
\item 
The tables and figures are ``floating'' items, hence may not appear at the location where they were coded by the users. The \LaTeX\ algorithm will decide the best place to display them, and this behavior can be adjusted/forced by float specifiers combinations (e.g., \texttt{[h, t, p, b, !, H]}). As the floats cannot be broken, they will usually be displayed on the top of the next page. With proper application of these float specifiers and moving the location of the float source code (moving text around), the desired effect can be achieved.  
\item
With \texttt{tblr} environment, the \texttt{H} is known to create increased row spacing (possible compatibility issue between \texttt{tabularray} and \texttt{float} packages). Therefore, \texttt{[h]} or \texttt{[h!]} should be used instead to maintain the intended row spacing. This issue was not found with \texttt{tabular} environment. 
\item
Several of the user-developed shortcut commands utilized in the class can be achieved using the \LaTeX\ direct commands---as the shortcuts are simply \cmd{newcommand} with arguments defined in the class using direct \LaTeX\ commands.   
\item
URLs, especially the wrapping issue at the right edge, should be handled differently, whether they are in the body text or in the bibliography. While the body text is automatically handled by the class, the bibliography URLs are handled by using the penalty commands (refer to ``\texttt{NDSU-Thesis-Extended.tex}''; URL: \url{https://github.com/CIgathi/NDSU-Thesis-Class} for details). 
\end{itemize}

%--------------------------------------------------------------
\subsection{NDSU resources and information for disquisition writers} \index{resources}

\begingroup
\footnotesize
\begin{tabular}{p{1.2in} p{5in}}
\toprule
\textbf{NDSU website} & \textbf{For more information about:}\\
\toprule
Graduate School $>$ \par Current Students & Graduate School policy, deadlines, forms, graduation. \par {\small URL: \url{https://www.ndsu.edu/gradschool/current_students}}\\[4mm]
\midrule
Graduate School $>$ \par Dissertations, Theses, \par and Papers & Disquisition policies, formatting guidelines, templates, a guide to using the templates, submission procedures, publication, and delayed release information. \par {\small URL: \url{https://www.ndsu.edu/gradschool/current_students/graduation/papers_theses_dissertations}}\\[4mm]
\midrule
Center for Writers \par & Writing consultations, writing workshops, disquisition boot camps, writing resources on site and online. \par{\small URL: \url{https://www.ndsu.edu/cfwriters/}}\\[4mm]
\midrule
Thesis Formatting \par Services---MS Word & The Instructional Design Center (IDC), Microsoft Office \& Thesis Formatting Specialist provides technical training and problem-solving for thesis formatting. Walk-ins are welcome at the IDC in the Quentin Burdick Building (QBB), 150G. For complex issues, we ask that you call ahead to schedule an appointment with the Thesis Formatting specialist at 701-231-5130. \par{\small URL: \url{https://kb.ndsu.edu/page.php?id=115414}}\\[4mm]
\midrule
Formatting Tutorial \par Videos---MS Word & Hosted on the GPS Academy YouTube Channel. ``Navigating the Review Process'' workshop recording and ``Word Crash Course'' videos on In-Paragraph Formatting; Styles; Section Breaks and Page Numbers; Table of Contents; Tables and Figures; Landscape Pages and More about Tables; Template Overview. \par{\small URL: \url{https://youtube.com/playlist?list=PLWKsN3lW50IexOpLRkNjh2HDuVKp3XjFH&si=Cksb4ZlFK3X1uCCh}}\\[2mm]
&\hfill {\scriptsize \textcolor{gray}{Links accessed on 22 Sep. 2024}} \\
\bottomrule
\end{tabular}
\endgroup

%--------------------------------------------------------------
\subsection{Limitations of the class} \index{limitations of class}
\begin{itemize}
\item \emph{Shortcuts for tables}: Unlike the figure environment, tables vary a lot (e.g., number of columns and rows), hence not possible to develop shortcuts (such as \cmd{myfig}\{\}). Therefore, no shortcuts were coded in the class, and the users should go for the basic command using the \cmd{table} environment. 

\item 
\emph{Citations in captions}: This means using \cmd{citep}\{\ldots\} or \cmd{citet}\{\ldots\} or \cmd{cite}\{\ldots\} inside the \\ \cmd{caption}\{\ldots\} argument of tables and figures is not supported. However, the output of the cites can be hard-coded (after noting the output outside of the caption) as a workaround. It should be noted that cite commands are known to work in captions of other classes. 

\item \emph{Limited packages loaded}:  Only limited packages that are thought to be useful for students and cater to most thesis/dissertations, in general, are included in the class. Therefore, for special needs, users need to include the respective packages in the preamble. 
\end{itemize}

%-------------------------------------------
%\enlargethispage{2\baselineskip}
\vspace{-1ex}
\subsection{Voice of the \TeX\ and \LaTeX\ developers!}
\vspace{-1ex}
It is interesting to know what the original developers of \TeX\ and \LaTeX\ have to say about this system of document preparation. Following are the quotes from the developers about how people feel, perceive, and use the system for their documentation needs.

\vspace{-1ex}
\begin{quote}
``\emph{I never expected \TeX\ to be the universal thing that people would turn to for the quick-and-dirty stuff. I always thought of it as something that you turned to if you cared enough to send the very best.}'' \hfill---\ix{Donald Knuth} (Developer of \TeX\  [on which \LaTeX\ is based])
\end{quote}

\vspace{-2.5ex}
\begin{quote}
``\emph{\LaTeX\ is easy to use---if you're one of the 2\,\% of the population who thinks logically and can read an instruction manual. The other 98\,\% of the population would find it very hard or impossible to use.}'' \hfill---\ix{Leslie Lamport} (Developer of \LaTeX)
\end{quote}

\vspace{-1ex}
It is safe to assume that students who came this far should have ``cared enough'' to improve the quality of their thesis/dissertation, and some who may think they are in the 98\,\% might discover that they have better logical skills than they originally believed. Furthermore, using \LaTeX\ for the documentation needs (e.g., thesis/dissertation, paper, report, book, letter, CV, and so on) should be considered a useful skill in itself that students can pick up and use throughout their career.

%--------------------------------------------------------------
\vspace{-2.5ex}
\section*{Acknowledgements}\index{acknowledgements}
\addcontentsline{toc}{section}{Acknowledgements}
\vspace{-1ex}
{\scriptsize 
The authors gratefully appreciate the leadership and inputs from Danjel Nygard, Dissertation and Thesis Coordinator, NDSU, Fargo, ND, in developing this ``ndsu-thesis-2022'' \LaTeX\/ class. The support and involvement of James Thorne, Ph.D. Student, Department of Mathematics, NDSU, Fargo, ND, the maintainer of the previous version of this class in CTAN, is also highly appreciated.}
%\enlargethispage{\baselineskip}
%--------------------------------------------------------------

%\newpage
%\vspace{-1.5ex}
{%\small 
\footnotesize
%\setlength{columnsep}{3cm}
\vspace{-1ex}
\printindex
}
%\enlargethispage{6\baselineskip}
%--------------------------------------------------------------

\vspace{4ex}


\begin{center}
{\Large\bfseries
\LaTeX{} NDSU-thesis-2022 class
}
\vspace{4ex}

---  $ \star \star \star$  ---
\end{center}


%\section{To be updated in the documentation---solution already found!}
%
%\begin{itemize}
%\item The new method of dealing with listings
%\item The word \textbf{\emph{disquisition}} in the second page
%\item Top 1 inch margin with the title -- the effect of single and 3-line titles
%\item Using myheading for unpublished papers---only published papers will have the footnote and the use of mypaperheading 
%\item Top 1 inch alignment of figures and tables when they are small and single on a page
%\item Use of return command (double backslash) to force bib urls that gets into the right 1 inch margin
%\item Bib file cleaning tool---\url{https://github.com/FlamingTempura/bibtex-tidy}. Go to this GitHub page and click on the \textcolor{RoyalBlue}{\textsf{\textul{Try it out}}} link below the list of files (as this link is too long to be included) 
%\item The sequence of 4 operations during compilation---especially for the development of the reference
%\item The necessity of removal of aux files after an unsuccessful compilation
%\item Use of block comment or barebone document for trying things out
%\end{itemize}


\end{document}

%------------------------------------------------------------------------------
%------------------------------------------------------------------------------
%-----------------------------------------------------------------------------
