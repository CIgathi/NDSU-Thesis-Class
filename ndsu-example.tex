%******************* START *******************
\documentclass[ms-thesis,12pt,mathdesign]{ndsu-thesis-2022}

%Refer documentation (ndsu-thesis-2022-documentation.pdf) for various options and commands

%******************* Packages, newcommands, and other customization *******************
\usepackage[style=apa,natbib=true,backend=biber]{biblatex}% works with \citep and \citet commands
\addbibresource{mybib.bib}% *.bib extension is necessary 
\renewcommand\myspacing{1.9} % 23 lines/page needs 1.9 for thesis

%***** SI units setup and a few custom unit commands *****
\sisetup{group-separator = {,}}% thousands separator comma; space default
\DeclareSIUnit\gal{gallon}
\DeclareSIUnit\ft{ft}
\newcommand{\sqft}[1]{\qty{#1}{foot}$^2$\xspace}% have math outside of SI
\newcommand{\cuft}[1]{\qty{#1}{\cubic\ft}\xspace}% SI standard commands
 
%******************* First and second page material *******************
\title{The Title of My M.S. Thesis}
\author{Samuel Fargo Bison}
\date{December 2023}
\progdeptchoice{Department} % Use Department (or) Program
\department{Mathematics}

\cchair{Prof. John Adams} % Use actual committee members names 
\cmembera{Prof. Abraham Lincoln}
\cmemberb{Prof. George Washington}
\cmemberc{Prof. Theodore Roosevelt} % If 3rd not required - delete this line 
\approvaldate{12/14/2022}
\approver{Prof. James Garfield}

%******************* Front matter *******************
\abstract{This is the abstract for my thesis. \\ \emph{Abstracts for doctoral dissertations must use 350 words or less. Abstracts for master's papers or master's theses must use 150 words or less.}\\ \kant[16]}% dummy text

\acknowledgements{I acknowledge people here. \\ \emph{Acknowledgements text should be placed here.} \\ \kant[15]}

\dedication{This thesis is dedicated to my cat, Mr. Fluffles.\\ \emph{This section dedicates the disquisition to a few significant people. The text must be double-spaced and aligned center to the page.} \\ Which is already taken care of by this \LaTeX\ class.}

\preface{You can put a preface here. \\ \emph{This section is optional!} \\ \kant[14]}

\listofabbreviations{% may use title case
AC          	& alternating current \\
NDSU     	& North Dakota State University \\
ZL       	& zeta level}

\listofsymbols{% may use sentence case
$A$     	& area (\unit{\m\squared})\\
$e$     	& Euler's constant (\num{2.718281828}) \\
$R^2$   	& coefficient of determination}

%******************* Document start *******************
\begin{document}

%******************* First chapter - paper style *******************
\mypaperheading{The First Chapter - Paper Style - Long title of this technical paper}{This paper is planned to be submitted as a peer-reviewed article \ldots\ more information about the author(s),  title,  \emph{journal},  to be added.}

%------------------------------------------------------
\section{Abstract}
Paper-styled chapters will have abstracts. Abstract of this chapter goes here. \kant[1]

%------------------------------------------------------
\section{Section ($\Rightarrow$ 1st level; Title Case; Centered; Boldface)}
This is the first section of the thesis (1st level: 1.2. Section). \kant[2]

%------------------------------------------------------
\section{Section}
This is the second section of the thesis (1st level: 1.3. Section). \kant[3]

%------------------------------------------------------
\subsection{Subsection ($\Rightarrow$ 2nd level; Title Case; Left-justified; Boldface)}
This is the subsection text (2nd level: 1.3.1. Subsection). \kant[4]

%------------------------------------------------------
\subsubsection{Subsubsection ($\Rightarrow$ 3rd level; Title Case; Left-justified; Boldface; Italics)}
This is the subsection text (3rd level: 1.3.1.1. Subsubsection). \kant[5]

%------------------------------------------------------
\paragraph{Paragraph ($\Rightarrow$ 4th level; Sentence case; Left-justified; No bold; Italics)}
This is the subsection text (4th level: 1.3.1.1.1. Paragraph). \kant[6]

%------------------------------------------------------
\subparagraph{Subparagraph ($\Rightarrow$ 5th level; Sentence case; Left-justified; No bold; Regular)}
This is the subsection text (5th level: 1.3.1.1.1.1. Paragraph). \kant[7]

%------------------------------------------------------
\section{Table and Figure}
This is the third section of the thesis (1st level: 1.4. Section). This section illustrates the inclusion of a simple table (\cref{tab:1}) and a figure shown later.

\begin{table}[h]
\centering
\caption{Table captions go at the top of the table. This was a long caption of the table included in the first chapter --- so that we see how it breaks into another line and has a single spacing. Usually, tables are of full-width and are demonstrated subsequently.}
\vspace{-1ex}
\begin{tabular}{clr}
\toprule
Number & Month & Days\\
\midrule
\#1 & January    & 31\\
\#2 & February   & 28\\
\#3 & March      & 31\\
\bottomrule
\end{tabular}
\label{tab:1}
\end{table}	\kant[7]

Now the figure (\cref{fig:1}) illustrates an example figure from the \texttt{mwe} package.

\myfig{H}{0.525}{example-image-duck}{Caption for this example image in this first chapter.}{fig:1}	
\kant[8-9]

\myfig{H}{0.525}{example-image-duck}{Caption for this example image in this first chapter.}{fig:1}	
\kant[8-9]

%******************* Second chapter - regular *******************
\myheading{The Second Chapter - Regular Style - Long title for this chapter}

Regular style chapters will not have abstracts. General information or outline of the chapter is given here --- before breaking into sections. 

%------------------------------------------------------
\section{Excellent Results}
This is another section of the thesis (1st level: 2.1. Experimental Results).\Cref{tab:2} presents the results in a tabular form that spans the entire width. Please note the results shown (\cref{tab:2}) are preliminary.

\begin{table}[ht]
\centering
\caption{Table spanning entire width (full-width) using \texttt{setlength} and
\texttt{tabcolsep}.}
\vspace{-1ex}
\setlength{\tabcolsep}{3.75em}
\begin{tabular}{@{\hspace{2ex}} lccr @{\hspace{2ex}}}
\toprule
Number & Name of month & Days & Season\\
\midrule
\#4 	& April  & 30		& Spring\\
\#5 	& May    & 31		& Summer\\
\#6 	& June   & 30		& Summer\\
\bottomrule
\end{tabular}
\begin{tablenotes}[flushleft]
\item \hspace{-1ex} Note: The \texttt{tablenotes} environment produces table footnotes. 
\end{tablenotes}
\label{tab:2}
\end{table}	\kant[7-8]

%------------------------------------------------------
\subsection{Minor Results}
This is a subsection of the thesis (1st level: 2.2. Experimental Results). 	

\kant[8]
The \Cref{fig:2} is an example image with command showing all arguments including the optional caption placement. The example figure (\cref{fig:2}) is included in the \texttt{mwe} package.

\myfig[2ex]{H}{0.35}{example-image}{Caption for this example image demonstrating an optional 2ex vertical spacing. Compare this with a narrow caption spacing without optional argument in 
\cref{fig:1}.}{fig:2}
\kant[8]

%------------------------------------------------------
\section{Equations}
\kant[2]

\myeqn{% shortcut for equation vertically spaced
y = (mx + c) \times \text{NCF} \times S_\text{factor} \times c_p \times M_\text{p}
\label{eq:lin}
}

\noindent where $y$ is the dependent variable, $m$ is the slope, $x$ is the independent variable, $c$ is the $y$ intercept, NCF is the normalized conversion factor, $S_\text{factor}$ is the scale factor, $c_p$ is the specific heat capacity at constant pressure ($p$, variable), and $M_\text{p}$ is the mass of a proton (p, descriptive). 

Note how variables, abbreviations, and subscripts are coded in \cref{eq:lin}. Refer Extended Thesis to know more about equations and shortcuts. 

%------------------------------------------------------
\section{Schemes}
\kant[2]

\begin{scheme}
\centering
\includegraphics[width=0.45\textwidth]{LampFlowchart}
\caption{Flowchart of controls of light bulb --- A scheme}
\label{sc1}
\end{scheme}
%
\kant[2]


\mysch[-2ex]{h!}{0.3}{LampFlowchart}{Caption for this example image demonstrating an optional 2ex vertical spacing. Compare this with a narrow caption spacing without optional argument in \cref{fig:2}.}{sc2}

\kant[2]\kant[9]

The (\cref{sc1,sc2}) are good. And, stating this differently all the \Cref{sc1,sc2,sc3} are good too. Note that \Cref{sc3} is in landscape mode.

\myschls[1ex]{p}{0.55}{LampFlowchart}{Landscape scheme --- Flowchart of controls of light bulb. Optional 2ex vertical spacing was used.}{sc3}


%------------------------------------------------------
\section{Some References}
Referring to all entries in the ``\texttt{mybib.bib}'' file to generate the citations here and the listing using the \texttt{\textbackslash citep\{\ldots\}} ``natbib'' command (cite parenthesis \citep{texbook,lcompanion,latex2e,knuth1984,lesk1977,amsthm2017,calvo2004using,cannayen2011latex,kopka2004guide,notso2021,bari2016identification}.

The same using \texttt{\textbackslash citet\{\ldots\}} command (cite text) in the running text as: The authors \citet{texbook,lcompanion,latex2e,knuth1984,lesk1977,amsthm2017,calvo2004using,cannayen2011latex,kopka2004guide,notso2021,bari2016identification} have something to do with \LaTeX. For most bibliography citations and list creation, these two commands are sufficient.



%------------------------------------------------------
\subsection{Appendix A Subsection}
This is an example long table.
\vspace{2ex}
\setlength\LTleft{0pt}
\setlength\LTright{0pt}
 
\begin{landscape}
{\small
{\renewcommand{\arraystretch}{0.6}
\begin{ThreePartTable}
  \begin{TableNotes}
  \baselineskip=0.5\baselineskip
    \item[] $\dag$ MD - Methods distance i.e. total polygonal distance of all methods taken in the selected order   
    \item[] $\ddag$ TSP - Traveling salesperson distance i.e., total polygonal distance of all methods following traveling salesman technique; Origin was the outlet location where bales were finally transported; and medoid was the aggregation method where it coincided on one of the field stacks but other methods may not.
  \end{TableNotes}
  \begin{longtable}{lll lll ll rrr}
  \caption{\normalsize A long table - spanning 3 pages - an example taken from our research group work on ``Methods of optimum bale stack locations and their logistics distances and methods combined distances.''}\label{tab1}\\[-2ex]    
%  \tabletopinfols% defined in the preamble
    \endfirsthead
   
 \multicolumn{11}{l}%
{{\normalsize\tablename\ \thetable{} Methods of optimum bale stack locations and their logistics distances  -- (\emph{continued}).}} \\[2ex]
%\tabletopinfols% defined in the preamble
    \endhead
   
    \cmidrule{10-11}
    \multicolumn{11}{r}{\textit{continued \ldots}}
    \endfoot
    \bottomrule
    \insertTableNotes
    \endlastfoot
       
% the contents of the table
0.41 & 3 & Origin  & 0.196 & 0 & 0.196 & 0.070 & 0.045 & 123 & 234 & 345\\
$[1]$ &  & Field middle  & 0.085 & 0.045 & 0.130 \\
&  & Middle data range  & 0.070 & 0.061 & 0.131 \\
&  & Centroid & 0.068 & 0.062 & 0.130 \\
&  & Geometric median & 0.065 & 0.064 & 0.129 \\
&  & Medoid  & 0.068 & 0.075 & 0.143 \\
\midrule
0.51 & 4 & Origin  & 0.240 & 0 & 0.240 & 0.054 & 0.048 & 123 & 234 & 345\\
$[1.25]$ &  & Field middle  & 0.107 & 0.050 & 0.158 \\
&  & Middle data range  & 0.108 & 0.052 & 0.160 \\
&  & Centroid & 0.102 & 0.057 & 0.159 \\
&  & Geometric median & 0.099 & 0.067 & 0.166 \\
&  & Medoid  & 0.101 & 0.072 & 0.172 \\
\midrule
1.01 & 8 & Origin  & 0.462 & 0 & 0.462 & 0.095 & 0.051  & 123 & 234 & 345\\
$[2.5]$ &  & Field middle  & 0.404 & 0.142 & 0.546 \\
&  & Middle data range  & 0.205 & 0.109 & 0.315 \\
&  & Centroid & 0.206 & 0.114 & 0.320 \\
&  & Geometric median & 0.205 & 0.109 & 0.314 \\
&  & Medoid  & 0.206 & 0.103 & 0.308 \\
\midrule
2.02 & 18 & Origin  & 1.80 & 0 & 1.80 & 0.054 & 0.034 & 123 & 234 & 345 \\
$[5]$ &  & Field middle  & 0.87 & 0.30 & 1.17 \\
&  & Middle data range  & 0.87 & 0.30 & 1.17 \\
&  & Centroid & 0.86 & 0.31 & 1.17 \\
&  & Geometric median & 0.86 & 0.31 & 1.18 \\
&  & Medoid  & 0.89 & 0.35 & 1.24 \\
\midrule
4.05 & 33 & Origin  & 5.26 & 0 & 5.26 & 0.144 & 0.100 & 123 & 234 & 345 \\
$[10]$ &  & Field middle  & 3.11 & 0.85 & 3.96 \\
&  & Middle data range  & 3.11 & 0.86 & 3.97 \\
&  & Centroid & 3.11 & 0.86 & 3.97 \\
&  & Geometric median & 3.11 & 0.88 & 3.99 \\
&  & Medoid  & 3.45 & 1.09 & 4.53 \\
\midrule
8.09 & 67 & Origin  & 14.63 & 0 & 14.63 & 0.024 & 0.021 & 123 & 234 & 345 \\
$[20]$ &  & Field middle  & 7.29 & 2.41 & 9.71 \\
&  & Middle data range  & 7.29 & 2.43 & 9.72 \\
&  & Centroid & 7.29 & 2.43 & 9.72 \\
&  & Geometric median & 7.28 & 2.45 & 9.73 \\
&  & Medoid  & 7.29 & 2.41 & 9.70 \\
\midrule
16.19 & 133 & Origin  & 40.67 & 0 & 40.67 & 0.074 & 0.072 & 123 & 234 & 345 \\
$[40]$ &  & Field middle  & 20.28 & 6.54 & 26.82 \\
&  & Middle data range  & 20.29 & 6.61 & 26.89 \\
&  & Centroid & 20.28 & 6.51 & 26.79 \\
&  & Geometric median & 20.28 & 6.58 & 26.86 \\
&  & Medoid  & 20.52 & 6.88 & 27.39 \\
\midrule
32.38 & 270 & Origin  & 117.89 & 0 & 117.89 & 0.060 & 0.052 & 123 & 234 & 345 \\
$[80]$ &  & Field middle  & 58.92 & 18.11 & 77.03 \\
&  & Middle data range  & 58.92 & 18.22 & 77.14 \\
&  & Centroid & 58.92 & 18.16 & 77.08 \\
&  & Geometric median & 58.92 & 18.19 & 77.11 \\
&  & Medoid  & 59.18 & 18.11 & 77.29 \\
\midrule
64.75 & 540 & Origin  & 333.12 & 0 & 333.12 & 0.049 & 0.043 & 123 & 234 & 345 \\
$[160]$ &  & Field middle  & 166.52 & 51.21 & 217.73 \\
&  & Middle data range  & 166.53 & 51.41 & 217.93 \\
&  & Centroid & 166.52 & 51.26 & 217.78 \\
&  & Geometric median & 166.52 & 51.30 & 217.82 \\
&  & Medoid  & 166.81 & 51.23 & 218.05 \\
\midrule
129.5 & 1082 & Origin  & 943.38 & 0 & 943.38 & 0.051 & 0.029 & 123 & 234 & 345 \\
$[320]$ &  & Field middle  & 470.83 & 145.65 & 616.48 \\
&  & Middle data range  & 470.83 & 145.79 & 616.62 \\
&  & Centroid & 470.83 & 145.91 & 616.74 \\
&  & Geometric median & 470.83 & 145.83 & 616.66 \\
&  & Medoid  & 471.26 & 148.53 & 619.79 \\
\midrule
259 & 2163 & Origin  & 2665.34 & 0 & 2665.34 & 0.028 & 0.027 & 123 & 234 & 345 \\
$[640]$ &  & Field middle  & 1331.20 & 410.81 & 1742.01 \\
&  & Middle data range  & 1331.21 & 411.45 & 1742.66 \\
&  & Centroid & 1331.19 & 411.07 & 1742.27 \\
&  & Geometric median & 1331.19 & 411.25 & 1742.44 \\
&  & Medoid  & 1331.32 & 407.51 & 1738.83 \\
\midrule
517 & 4324 & Origin  & 7531.35 & 0 & 7531.35 & 0.022 & 0.020 & 123 & 234 & 345 \\
$[1280]$ &  & Field middle  & 3765.75 & 1160.34 & 4926.09 \\
&  & Middle data range  & 3765.77 & 1160.95 & 4926.72 \\
&  & Centroid & 3765.75 & 1160.51 & 4926.26 \\
&  & Geometric median & 3765.75 & 1160.39 & 4926.15 \\
&  & Medoid  & 3765.86 & 1159.71 & 4925.57 \\
\label{longtabls} % label inside the innermost longtable environment
\end{longtable}
\end{ThreePartTable}
}
}
\end{landscape}
 
\setlength{\parindent}{0.5in}
\vspace{-2ex}
 
%-----------------------------------------------------------------------------




%******************* Bibliography handling *******************
\makerefs %For individual chapter references - command should be inside refsection environment


%******************* Named appendix A *******************
\namedappendices{A}{Named first appendix}
Appendix material can be included here. First a paragraph of text and then an example figure (fig.~\ref{fig:ap1}).

%------------------------------------------------------
\section{Appendix A - Section With Figure}
\kant[9]
\myfigap{H}{0.5}{example-image-golden}{A golden ratio rectangle image.}{fig:ap1}	\kant[8]

\section{Appendix A - Section With Table}
And, then including a table (table.~\ref{tab:ap1}).

\begin{appendixtable}[h]
\centering
\caption{Use of \texttt{tblr} environment for full-width table - applicable to both main text 
and appendix.  Note the use of \texttt{booktabs} commands and `X' parameters to reproduce 
Table~\ref{tab:2}.}
\begin{tblr}{*4X}
\toprule
Number 	& Name of month 	& Days 	& Season\\
\midrule
\#7 			& July       		& 30 		& Spring\\ \cmidrule[lr]{2-4}
Multicolumn 	&\SetCell[c=3]{c} The three columns combined \\ \cmidrule[lr]{2-4}
\#8 			& August 		 & 31 	& Summer\\
\#9 			& September 	& 30 		& Summer\\
\bottomrule
\end{tblr}
\begin{tablenotes}[flushleft]
\item \hspace{-1ex} Note: The \texttt{tablenotes} environment produces table footnotes.  
Refer to \texttt{tabularray} documentation for further details.  
\end{tablenotes}
\label{tab:ap1}
\end{appendixtable}

For other types of tables and figures, such as landscape tables, long tables, landscape long tables, landscape figures, subfigures, subfigures spanning multiple pages, and multiple figures in landscape see NDSU-Thesis-Extended code and output. 

%------------------------------------------------------
\subsection{Appendix A Subsection}
\kant[10]


%******************* Named Appendix B *******************
\namedappendices{B}{Named second appendix}
Appendix material can be included here. First including a figure (fig.~\ref{fig:ap2}).

%------------------------------------------------------
\section{Appendix B - Section With Figure}
\kant[9]
\myfigap[0.5ex]{H}{0.6}{example-grid-100x100pt}{A $10 \times 10$ grid of different concentric colors.}{fig:ap2}

%------------------------------------------------------
\section{Appendix B - Section With Table}
Now coding another appendix table (table.~\ref{tab:ap2}) that spans the entire width using the manual method (using `tabcolsep' command; and `resize' command to fit large tables).

\begin{appendixtable}[h]
\centering
\caption{Squares and cubes in named appendix table using \texttt{siunitx} and \texttt{tabularray} 
packages.}
\begin{tblr}{X X[c] X[r] X[1.5,r]}
\toprule
Number & Square        & Cubes          & Fourth power\\
\midrule
11 	   & 121   			        & \num{1331} 		   & \num{14641}\\
22 	   & 484  			        & \num{10648}		   & \num{234256}\\
333 	   & \num{110889}             & \num{36926037}	   & \num{12296370321}\\
\bottomrule
\end{tblr}
\label{tab:ap2}
\end{appendixtable}
 
%------------------------------------------------------
\subsection{Appendix B Subsection}
\kant[11]

\closeappendices  % Refer documentation Table 3 for proper closing 

\end{document}
%******************* END *******************