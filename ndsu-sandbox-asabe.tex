%******************* START *******************
\documentclass[12pt,mathdesign]{ndsu-thesis-2022}

%----------------------------------------------------
\usepackage[style=apa,natbib=true,backend=biber]{biblatex}
\addbibresource{mybib.bib} 
\newcommand\myspacing{1.9} % 23 lines/page needs 1.9 for thesis
%----------------------------------------------------

\graphicspath{{./figures/}}
\setbiblatexASABE  % shortcut to setup ASABE references
\setcapASABE         % shortcut to setup ASABE captions

\newcommand\makerefs{
\printbibliography[heading=subbibnumbered, title={References}]
} % shortcut - section numbered; subbibintoc - unnumbered


\title{My NDSU Thesis --- Sandbox}


%************ Document start ************
\begin{document}
\begin{spacing}{\myspacing}      % New line spacing - 23 lines per page

%----------------------------------------------------
\myheading{Test Chapter for NDSU Thesis Class Sandbox}

This ``\texttt{ndsu-sandbox.tex}'' file can be used as a sandbox to try out things in the actual NDSU thesis environment. \textcolor{gray}{Things tested here (including the bibliography) can be readily inserted into the original thesis/dissertation document. Therefore, this lightweight source will be convenient to test things out.} So, go for it --- and remember anything is possible by \LaTeX{} (almost!?).

%----------------------------------------------------
\section{Section}
\subsection{Sub-Section}
\subsubsection{Sub-Sub-Section}

\textcolor{gray}{Dummy text from kantlipsum[9]. Reference listing on the next page. Check it for the intended formatting.} I refer to \citep{lamport94,kopka2004guide,baczkowski1990ndsu}. \kant[9]


\begin{table}[ht]
\centering
\caption{Professional looking fixed-width table using 
\texttt{booktabs} package.}
\begin{tabular}{ l c r }
\toprule
Number & Our rating & Month \\
(left) & (center)   & (right)\\
\midrule
1 & Colder & January \\
2 & Okay   & February \\
3 & Good   & March\\
\bottomrule
\end{tabular}
\label{tab22}
\end{table}

\kant[9]

\begin{table}[h!]
\centering
\caption{Professional looking automatic full-width table using \texttt{tblr} environment and \texttt{booktabs} package.}
\begin{tblr}{X | X[c] | X[r]}
\toprule
Number & Our rating & Month \\
(left) & (center)   & (right)\\
\midrule
1 & Colder & January \\
2 & Okay   & February \\
3 & Good   & March\\
\bottomrule
\end{tblr}
\label{tab25}
\end{table}

\kant[9]

\myfig{H}{0.4}{frog.jpg}{Figure short caption is centered. 
Use of myfig command.}{fig2}


\kant[9]
\myfig{H}{0.4}{frog}{Figure short caption is centered. 
Use of myfig command; Now long caption that will be left-justified.}{fig3}

%----------------------------------------------------
\kant[2-4]
\makerefs % for biblatex 

%----------------------------------------------------
%----------------------------------------------------
\myheading{Test Second Chapter for NDSU Thesis Class Sandbox}
\begin{refsection}
\kant[20-21]

%----------------------------------------------------
\section{Section}
\subsection{Sub-Section}
\subsubsection{Sub-Sub-Section}

\textcolor{gray}{Dummy text from kantlipsum. Reference listing on the next page. Check it for the intended formatting.} I refer to \citep{butin2009education, rudestam2014surviving, Goossens2008g}. \kant[9]

\myfig{H}{0.4}{frog}{Short caption is centered. Use of myfig command.}{fig4}

\kant[10]
\myfig{H}{0.95}{fig-LOS}{Figure from the figures folder. Short caption is centered. Use of myfig command; Now long caption that will be left-justified.}{fig5}

%----------------------------------------------------
\kant[2-5]
\newpage
\makerefs % for biblatex 
\end{refsection}
%----------------------------------------------------


\end{spacing}
\end{document}
%******************* END *******************